\documentclass{article}
\usepackage[utf8]{inputenc}
\usepackage{todo}

\title{Thesis Introduction and Background}
\author{Haley Brandt-Erichsen}
\date{November 2015}

\begin{document}

\begin{titlepage}
    {\centering
    Rethinking American Negotiation Strategies in North Korean Nuclear Talks \par
    \vspace{0.5cm}
    by \par
    \vspace{0.5cm}
    Haley Brandt-Erichsen \par
    \vspace{0.5cm}
    Submitted to the \par Department of Nuclear Science and Engineering \par in Partial Fulfillment of the Requirements for the Degree of \par
    \vspace{0.5cm}
    Bachelor of Science \par
    \vspace{0.5cm}
    at the \par
    \vspace{0.5cm}
    Massachusetts Institute of Technology \par
    \vspace{0.5cm}
    June 2016 \par
    \vspace{0.5cm}
    \copyright 2016 Haley Brandt-Erichsen \par All rights reserved \par
    \vspace{0.5cm}
    The author hereby grants to MIT permission to reproduce and to
distribute publicly paper and electronic copies of this thesis document in whole or in part
in any medium now known or hereafter created. \par
    }
    \vfill
    
    {Signature of Author: \_\_\_\_\_\_\_\_\_\_\_\_\_\_\_\_\_\_\_\_\_\_\_\_\_\_\_\_\_\_\_\_\_\_\_\_\_\_\_\_\_\_\_\_\_\_\_\_\_\_\_\_\_\_\_\_\_\_\_\_\_\_\_\_\_\_}
    \begin{flushright}
        Department of Nuclear Science and Engineering \par 
        DATE GOES HERE
    \end{flushright}
    \vspace{0.5 cm}
    {Certified by: \_\_\_\_\_\_\_\_\_\_\_\_\_\_\_\_\_\_\_\_\_\_\_\_\_\_\_\_\_\_\_\_\_\_\_\_\_\_\_\_\_\_\_\_\_\_\_\_\_\_\_\_\_\_\_\_\_\_\_\_\_\_\_\_\_\_\_\_\_\_\_\_\_\_\_\_\_\_\_}
    \begin{flushright}
        R. Scott Kemp \par
        Assistant Professor of Nuclear Science and Engineering \par
        Thesis Supervisor
    \end{flushright}
    \vspace{0.5 cm}
    {Accepted by: \_\_\_\_\_\_\_\_\_\_\_\_\_\_\_\_\_\_\_\_\_\_\_\_\_\_\_\_\_\_\_\_\_\_\_\_\_\_\_\_\_\_\_\_\_\_\_\_\_\_\_\_\_\_\_\_\_\_\_\_\_\_\_\_\_\_\_\_\_\_\_\_\_\_\_\_\_\_\_}
    \begin{flushright}
        Michael P. Short \par
        Assistant Professor of Nuclear Science and Engineering \par
        Chairman, NSE Committee for Undergraduate Students
    \end{flushright}
    
    
\end{titlepage}

\newpage
\thispagestyle{empty}
       \addtocounter{page}{-1}
       \null
\newpage

\tableofcontents

\pagebreak

\section{Acronyms Used}

DPRK – Democratic People’s Republic of Korea (North Korea)

\noindent IAEA – International Atomic Energy Agency

\noindent KCNA – Korean Central News Agency (North Korean state-owned news source)

\noindent KEDO – Korean Peninsula Energy Development Organization

\noindent LWR – light-water reactor

\noindent NPT - Treaty on the Non-Proliferation of Nuclear Weapons (Non-Proliferation Treaty)

\noindent ROK – Republic of Korea (South Korea)

\noindent UNSC – United Nations Security Council

\section{Introduction}

\subsection{Motivation}

The specter of North Korean nuclear weapons has haunted the world since the early days of their nuclear reactor program in the 1960s \cite{pincus}. The international community, in particular the United States and South Korea, has always been adamant that a nuclear-armed North Korea is intolerable \cite{kerry,parksk}.

In the past, many believed that it was possible to achieve peaceful nuclear power for the DPRK without undue risk of weapon production – indeed, the 1994 Agreed Framework outlined a plan in which the United States would help North Korea develop a light-water reactor program \cite{agreed}. However, in recent years, even “proliferation-resistant” reactors like the LWRs have been deemed too great of a risk. In 2008, Lee Myung-bak ran his presidential campaign in South Korea on the founding principle that aid to the North should be dependent upon complete denuclearization \cite{snyder}. As of late 2015, the United States held the official position that resuming multilateral negotiations with the DPRK could not occur until denuclearization was guaranteed \cite{pennington}. North Korea, meanwhile, is insistent that it has the right to nuclear materials for both power plants and weaponry \cite{kcna}. 

A variety of methods have been tried to resolve this disagreement, from sanctions and aid restrictions to mandated denuclearization as a condition on resumption of treaty talks \cite{bajoria,davenport}. These methods have, at best, resulted in temporary disarmament or a partial rollback of the DPRK’s nuclear programs – but, in every case, the changes were only temporary and North Korea eventually returned to developing its nuclear weapons technology \cite{davenport,nti15,iaea09}.

When these methods have not succeeded, the UN and individual countries attempting to change the DPRK’s behavior have not systematically altered their approach \cite{davenport,nti15}. This indicates that negotiators believe the strategies themselves are valid, and grounded in a correct understanding of the DPRK’s nuclear situation. It is worthwhile to investigate whether this assumption is warranted.

\subsection{Objectives}

It is hypothesized that United States strategy in negotiating with North Korea is inconsistent with the technical facts of the North Korean nuclear program and past negotiating behavior. In other words, the cyclical nature of negotiations persists because negotiators base their positions on what they want to be true about the DPRK, rather than the facts of the situation. 

In order to test this hypothesis, data will be gleaned from analysis along three axes. Examination of the history of negotiations past will be combined with calculations on nuclear material stockpiles, weapons testing yields, and production capacity, as well as analysis of the rhetoric of press releases and offifical government statements. This combines qualitative and quantitative analysis in order to form the most complete picture of the DPRK's objectives and abilities possible.

A number of likely indicators of problems in American negotiation strategies can be predicted even before significant analysis has taken place. If North Korea's technical achievements are n line with or on trajectory to match its stated end goals for a nuclear arsenal and infrastructure, it seems unlikely to acquiesce to external pressures to reverse course. Also, if its rhetoric regarding denuclearization has been consistent across a wide variety of incentives and penalties, it is unlikely that variation in the terms of agreements along existing strategies will have a noticeable impact.

Meanwhile, there are also some indicators that American negotiation strategies hold the potential for success: if, for example, there have been previous agreements which, even if ultimately unsuccessful (as most if not all have eventually been), provided some sort of incremental improvement to the situation that lasted beyond the agreement's end. Also, if North Korea's technical capacity is still far from their desired endpoint and seems unlikely to reach it for many years, there is likely to be more room for productive negotiations.


\section{Background}

\subsection{Brief history of North Korean nuclear programs and negotiations}

Throughout the 1950s, the DPRK and the USSR worked together to initiate the North Korean nuclear power program, beginning construction on the Yongbyon nuclear complex in the early 1960s \cite{nti15}. The country became party to limited IAEA regulations in the following decade for the 5 MWe reactor constructed at Yongbyon, but did not develop a complete safeguards agreement until some time after its entrance into the NPT in 1985 \cite{iaea09}. During the 80s, North Korea continued to expand the Yongbyon complex and look into LWR technology \cite{ntiYongbyon}.

Over the next several decades, the US and the DPRK signed a number of agreements under which assurances were made about the provision of LWRs to North Korea in exchange for regulations intended to limit risk of proliferation \cite{agreed, davenport}. However, KEDO – the organization created to implement these agreements by constructing the LWRs – faced significant challenges in terms of funding and political support, eventually functionally collapsing under the strain of disagreements between the DPRK and member states in 2006 \cite{kedohist}.

Simultaneously, the Yongbyon reactors that the KEDO project was intended to replace existed in a constant state of flux. The complex would be shut down pursuant to some new agreement, or the IAEA would be allowed to inspect portions of the site…but then, a few years later, the inspectors would be expelled and the facilities restarted again \cite{davenport,nti15,iaea09}.

As the reprocessing facility in Yongbyn operated in fits and starts, the DPRK gradually accumulated enough plutonium to produce nuclear weapons. An underground nuclear test in a DPRK facility in 2006 shocked and horrified the international community, which immediately responded with political admonishments and economic sanctions \cite{davenport}. To date, the DPRK has tested nuclear weapons twice more, along with a number of missile tests going back to 1998 \cite{orfall} and including launches in 2013 that culminated in the successful placement of a satellite into orbit \cite{davenport}. In addition to the nuclear tests, missile tests and rocket launches by the DPRK have been roundly criticized by the UN and many individual nations as attempts to develop effective nuclear missiles \cite{unApr12}.

\Todo{'Update background for 2016 changes'}

\subsection{Current Political Status}

The North Korean nuclear program has been shut down and restarted a number of times, pursuant to various agreements made and then broken between the DPRK and a number of negotiating bodies \cite{bajoria,davenport}. The DPRK insists that its peaceful program should be allowed to continue \cite{kcna2}, while the US and South Korea claim that even this is unacceptable \cite{lee} – and demand complete commitment to denuclearization before any further negotiations can even begin.

This hard-line position is informed by fact – the Yongbyon reactor, nominally a peaceful source of power, did produce the material that has been used in the DPRK’s weapons program \cite{hecker}. But the stance has proven unhelpful in getting North Korea to the table for negotiations – they claim that it is their right as a sovereign nation to pursue both peaceful and military nuclear technology, and that maintaining this right is a necessary deterrent against foreign aggression \cite{kcna,kcna2}.

Each side, then, is demanding as a precondition for negotiation the very action that the other side refuses to even consider. It is perhaps unsurprising that this has not been effective. The result has been a cycle of behavior that has gone on for several decades: aggression and posturing on North Korea’s part is followed by sanctions and demands by other nations. Tension escalates until one side or the other begins to call for a diplomatic solution, which either ends in stalemate or produces results that only last for a short time before the agreements are broken again \cite{bajoria, davenport}.

\subsection{Previous Analysis of the Problem}

Many analyses of the nature of the politics surrounding DPRK nuclear diplomacy exist, and several have concluded that the problem is cyclic in nature \cite{blair,cfr,fisher,gause,habib,jun}. In general, the issue tends to be modeled as a series of cycles comprised of initial posturing, followed by aggression and escalation, and concluding with reconciliation that, depending on the analyst, may or may not be identified as sincere.

When these analyses are politically-motivated \cite{blair, cfr}, their logic tends to begin with the initial proposition that demand for complete denuclearization can someday be fulfilled, if only the correct combination of diplomacy and coercion can be found that will force the DPRK to comply. More academic treatments of the problem \cite{habib,jun} vary widely in their approach. Jun \cite{jun} identifies problems with past approaches, but seems to conclude that increased coordination between negotiating bodies and oversight of agreement implementation may yet salvage the denuclearization agenda. Meanwhile, Habib \cite{habib} claims that North Korea’s cyclic behavior is inevitable due to the nation’s military-first ethos. He further postulates that the nuclear program is a key part of this ethos, and thus that the DPRK will never sincerely agree to give it up.

Though the concept of modeling US-DPRK relations as a series of cycles is well-supported, the form this model takes within the literature leaves something to be desired. Often, for the sake of simplicity, pithiness, or with the hope of revealing some greater structure within a complicated historical progression, the model will postulate that all important US-DPRK negotiations can be seen as a repetition of a single process, three to five steps long \cite{fisher,jun}. It is possible - and, indeed, likely - that this is an underfitting that loses more in its lack of subtlety than it gains in elegance.

There is a decently broad body of research focusing on rhetorical analysis of the North Korean nuclear issue. Several authors \cite{rich12, rich14, sin} have worked with automated content analysis to look at how references to countries and specific incidents correlate with references to nuclear programs in official North Korean news outlets.

Additionally, work on the shifts and nuances of American rhetoric in nuclear negotiations with the DPRK exists - albeit in somewhat less detail than the work done on KCNA documents. Bleiker \cite{bleiker} identifies the relationship between the rise and fall of "rogue state" terminology used by various American presidents and US-DPRK relations. However, it does not appear that any kind of large-scale data-aggregating study has been done on American official statements in the way that Rich \cite{rich12, rich14} and Sin \cite{sin} worked with North Korean documents.

\section{Historical Analysis}
\subsection{Methodology}

Identifying the timing and relationship between various events - like sanctions, treaty negotiations, and missile tests - may illuminate consistency that will allow conclusions to be drawn about shifts in North Korean diplomatic posture. These correlations may indicate consistently successful (or unsuccessful) strategies on the part of North Korea, and they may provide insight as to the efficacy of favored political tools of those who would change the DPRK's behavior.

In keeping with established literature, analysis is patterned on the notion that US-DPRK relations are cyclic in nature. However, unlike much previous work, no attempt is made to fit individual behavior cycles to specific models. Instead, a broad analysis of instances of escalatory behavior is undertaken. Since this does not rely upon fitting all categories of events into a single three-to-five-step process (as is common within historical analysis of this problem \cite{fisher, jun}), it may be possible to identify correlations that have previously been missed.

Thus, behavioral cycles are here defined as any instigating event followed by a chain of other events which are direct responses to some action earlier in the chain. This allows classification into individual cycles for ease of pattern analysis, while avoiding the preconceptions built into a model with defined steps.

While even a model of this generality does exclude some relevant details (not every significant event in US-DPRK diplomatic history fits within an escalatory cycle), the purpose of this analysis is not a complete encapsulation of information about negotiations. Rather, the intention is to narrow a complicated multilateral diplomacy problem to a more tractable case, by forming a timeline of significant events that will serve as the backbone for subsequent analysis of technical and rhetorical changes.

\subsection{Early development of the North Korean nuclear program}
In 1952, the Democratic People’s Republic of North Korea (henceforth DPRK) established an organization to begin research into nuclear technologies, the Atomic Energy Research Institute\cite{ntiAERI}. Four years later, with little progress made, they signed an agreement with the USSR in order to train Korean scientists\cite{nti15}. Then, in 1959, additional agreements gave birth to the Yongbyon research complex, as the USSR began to assist the DPRK with construction and materials for a research reactor in addition to technical training\cite{nti15}.

By the early 1970s, nuclear expertise in the DPRK had advanced to the point that they were no longer reliant upon the Soviet Union for reactor technology (and, indeed, had begun to make their own improvements on the IRT-2000 research reactor design)\cite{nti15}. However, the partnership was still strong, and the USSR began to provide the DPRK with assistance in developing plutonium reprocessing capacity\cite{nti15}.

A decade later, North Korea’s technical development continued to improve rapidly as they pursued technologies all along the nuclear fuel cycle – as well as several significantly larger reactors (5 MWe in 1979\cite{ntiYongbyon}, 50 MWe in 1986\cite{ntiYongbyon2}). Simultaneously, they began to pursue light water reactors (LWRs) - signing the Treaty on the Non-Proliferation of Nuclear Weapons on the condition that they be provided additional construction assistance\cite{nti15}.   

\subsection{Cycle 1}

In 1992, provisions of the IAEA Safeguards Agreement associated with membership in the NPT went into effect, and the DPRK filed an initial report declaring the contents of their nuclear inventory\cite{iaea92}. However, IAEA analysis suggested a notably higher level of plutonium should exist in DPRK waste streams and stockpiles than was declared, and requested access to waste sites in order to verify or disprove the existence of undeclared plutonium\cite{iaea09}. 

Rather than accede to this request, the DPRK refused access, claiming that the sites were military in nature and thus exempt from the Safeguards Agreement\cite{nti15,iaea09}. In response, the IAEA requested authorization from the UN Security Council (UNSC) to perform special inspections on those sites\cite{nti15}.
 
In response to the IAEA’s request, North Korea declared it was withdrawing from the NPT in order to protect “the supreme interests of its country”, effective 90 days after the declaration as per Article X of the treaty\cite{npt}. The United States hastily began bilateral talks, and the day before the withdrawal was to come into effect, the DPRK announced that it was suspending this action for at least until negotiations were completed\cite{nti15}. The result of these negotiations, announced in July 1993, was a joint statement between the United States and North Korea. In this statement, the United States agreed to “support the introduction of LWRs” into North Korea to replace their existing graphite-moderated reactors, while the DPRK agreed to “begin consultations with the IAEA on outstanding safeguards and other issues as soon as possible”\cite{hayes}. 

\subsection{Cycle 2}

In February of 1994, the agreement with the IAEA prescribed by earlier negotiations was finalized, and inspectors were allowed back into several of the DPRK’s nuclear facilities\cite{davenport}. However, the inspections that were permitted to occur were incomplete, as the DPRK insisted that only “continuity of safeguards” was required\cite{iaea09}. Under this paradigm, IAEA inspectors were only allowed into areas they had already been permitted to access – further, they were not allowed to take additional actions that might verify or disprove the existence of undeclared plutonium stores. This angered the IAEA Board of Governors, who believed that they could not accurately determine whether proliferation-related programs were occurring under such conditions\cite{davenport}.

In May, the IAEA’s concerns were realized, as the 5 MWe Yongbyon reactor’s fuel rods were removed without supervision and stored without preserving details of their previous locations within the core. In doing this, the DPRK made it impossible to usefully examine the fuel rods to look for evidence indicating that some had been removed for plutonium production when inspectors had not been present\cite{nobacksies}. This caused the IAEA Board of Governors to release a statement condemning the DPRK’s actions and suspending all non-medical assistance to the country by the IAEA\cite{iaea94}.

In response, the DPRK withdrew its membership from the IAEA. As a party to the NPT, the IAEA claimed it was still subject to the existing safeguards agreements – but North Korea disagreed, and refused to allow inspectors into its nuclear facilities\cite{iaea09}. Tensions continued to rise, to the point that the United States began seriously considering air strikes on Yongbyon\cite{jun}. Eventually, however, the crisis was defused as Jimmy Carter traveled to the DPRK and began negotiations that eventually culminated in the Agreed Framework\cite{nti15}. Under this agreement, the DPRK agreed to freeze the Yongbyon graphite reactors and its reprocessing program, while the US made more concrete guarantees regarding provision of LWR technology and energy aid to offset the impact of the reactor freeze\cite{agreed}. By November, the IAEA was able to confirm that operations on Yongbyon had ceased\cite{davenport}.

\subsection{Cycle 3}

In August of 1998, the DPRK launched a Taepodong rocket in a (likely unsuccessful) attempt to carry a satellite into orbit\cite{orfall}. Many nations denounces this as an unacceptable missile test, and were concerned by the advances in range and complexity of North Korean missile technology that it displayed\cite{orfall}. As a result, Japan suspended diplomatic talks and considered halting its funding for the Agreed Framework’s LWR program\cite{orfall}.

Negotiations between the United States and North Korea began again, but were largely unsuccessful – the US offered sanction relief in exchange for termination of the DPRK missile program, but the latter claimed that sanction relief was already part of the Agreed Framework and thus not a valid incentive for negotiation\cite{davenport}. In various iterations, negotiations continue until September 1999, when the DPRK agreed to temporarily refrain from conducting long-range missile tests in exchange for limited sanction relief\cite{davenport}.

\subsection{Cycle 4}

In late 2002, an American official in North Korea for talks mentioned a variety of concerns held by the US regarding the DPRK’s record on nuclear proliferation and human rights\cite{davenport}. He suggested that the DPRK should work on these issues in order to improve relations with the United States, which was taken by DPRK officials as “high handed and arrogant” policy which placed unreasonable unilateral demands on them\cite{kcna3}. In this already-tense environment, conditions worsened when the US official told North Korean representatives that the United States knew about a secret enrichment facility, then publicly claimed that the North Korean representatives confirmed its existence\cite{davenport}. 

The DPRK denied any such admission, but the United States (along with other countries involved in the LWR construction program) declared that sufficient evidence of violation of several treaties existed that they were suspending the energy aid negotiated under the Agreed Framework\cite{iaea09}. The IAEA attempted to gather more facts on the situation, in a manner regarded by the DPRK as “acting under the manipulation of the United States”\cite{hurriyet} – so, in response, the DPRK began removing IAEA seals, expelling inspectors, and generally restarting its graphite reactor program\cite{iaea09}.

The IAEA Board of Governors was extremely displeased with this turn of events, and released a statement condemning the DPRK’s behavior\cite{iaea03}. In response, the DPRK withdrew from the NPT, claiming that they could bypass the requisite 3-month waiting period because their withdrawal in 1993 had simply been temporarily suspended\cite{kcna4}. The IAEA expressed “deep concern” and referred the issue to the UNSC, which also expressed “concern”\cite{iaea09}. However, such concern did not stop the Yongbyon 5 MWe reactor from being restarted, which occurred in February 2003\cite{davenport}. Over the next several months, talks between the US, the DPRK, and China occurred to little effect as reactor operation and spent fuel reprocessing continued apace\cite{davenport}.

\subsection{Cycle 5}
In September of 2005, the United States froze North Korean funds in Banco Delta Asia, citing money-laundering concerns, association with drug trafficking, and suspected US currency counterfeiting\cite{davenport}. Banco Delta Asia was designated a “primary money-laundering concern” and the bank was prohibited from doing business in US dollars. Immediately, other banks around the globe began to refuse to do business with the DPRK, fearing similar reprisals\cite{greenlees}.

The next major round of six-party talks began shortly thereafter, and the DPRK delegation focused on the issue of the bank freeze to the exclusion of other issues\cite{greenlees}. As a result, little progress was made and the talks stalled. In early 2006, the US Treasury Department and DPRK officials discussed ways to resolve the Banco Delta Asia conflict, but remained at a stalemate - the DPRK would return to talks if the funds were unfrozen, but the US wanted to discuss issues related to the funds in multilateral negotiations\cite{greenlees}.

\subsection{Cycle 6}

During the summer of 2006, North Korea fired a number of missiles, including a long-range Taepodong-2. In response, South Korea (henceforth, the ROK) halted aid programs, Japan imposed sanctions, and the UNSC sanctioned missile-related technology\cite{greenlees}. Additionally, the UNSC resolution urged a return to the six-party talks and North Korea’s previous (voluntary) missile test moratorium\cite{unsc06}. The DPRK “vehemently denounce[d] and roundly refute[d]” the resolution, vowing to “bolster its war deterrent for self-defense” in whatever ways it saw fit\cite{kcna5}.

This threat was made good in October, when the DPRK conducted its first nuclear test\cite{nti15}. Though the test was likely not particularly successful, it still shocked the international community – the UNSC responded with additional sanctions and demands that North Korea roll back its nuclear program, avoid any further testing, and return to IAEA oversight\cite{unsc1718}. The six-party talks resumed in November in an apparent victory for diplomatic efforts, but due to lingering disagreements over Banco Delta Asia and North Korea’s unwillingness to work “unilaterally”, the talks concluded without result at the end of the year\cite{davenport}.

The talks resumed in February, 2007 – this time, resulting in substantive agreements. The DPRK promised to return to the NPT and IAEA surveillance, and to shut down and seal the Yongbyon reactors and other nuclear facilities. In exchange, they would receive a sizable food and energy aid package\cite{6pt07}. Enactment of this agreement was briefly stalled over continuing concerns related to the Banco Delta Asia funds, but the United States eventually agreed to unfreeze them and in return the DPRK began to shut down the Yongbyon facility under IAEA supervision\cite{davenport}.

Under another agreement from the most recent round of six-party talks, the DPRK was supposed to submit a declaration of its nuclear programs by the end of 2007\cite{6pt07pt2}. It did not do so, due (according to the US State Department) to “some technical questions”\cite{seanm}. The statement was finally released in June, and despite concerns about the completeness of the document, the United States announced its intention to remove the DPRK from the State Sponsors of Terror list, as well as removing some sanctions and other trade barriers\cite{nti15}.

When the United States failed to remove the DPRK from the state Sponsors of Terror list after the 45-day waiting period had expired, the DPRK announced that it would halt demolishment of its graphite reactors and was willing to begin construction again\cite{davenport}. By September 2008, the DPRK had asked the IAEA to remove its seals on the North Korean reprocessing plant, and IAEA officials warned that the DPRK intended to begin reprocessing material shortly\cite{iaea09}. The US hastily reopened negotiations, and reached an agreement whereby the State Sponsors of Terror delisting would occur in exchange for a return to disablement\cite{nti15}.

\subsection{Cycle 7}

Speculations about a North Korean missile launch began in February 2009. Several nations released statements to the effect that such a launch would violate a UNSC resolution and therefore the DPRK should not expect to be able to go forward without serious consequences\cite{davenport}. When the DPRK warned international organizations of the time and likely location of rocket stage splashdowns, the ROK announced that it was considering joining the Proliferation Security Initiative\cite{davenport}.

North Korea did indeed launch a rocket – likely a modified Taepodong-2 long-range missile, and allegedly for the purposes of launching a satellite\cite{davenport}. The UNSC released a statement condemning the launch and calling for a revisiting and strengthening of sanctions, while urging the DPRK to return to six-party talks\cite{unsc09}. In response, the DPRK withdrew from the six-party talks and a number of its agreements with the United States\cite{niksch}, removed IAEA safeguards and ejected inspectors\cite{iaea09}, and resumed construction on the mothballed 5 MWe reactor at Yongbyon\cite{nti15}.

In May, the DPRK conducted another nuclear test, likely somewhat more successful than its first\cite{nti15}. The UNSC convened an emergency meeting and released a Presidential Statement condemning North Korean nuclear tests and recommending the strengthening of sanctions\cite{unsc09p}. The ROK made good on its threat to join the Proliferation Security Initiative – which North Korea declared an act of war, voiding the Korean War armistice\cite{glionna}. The UNSC passed another resolution once again condemning North Korean nuclear tests and strengthening arms embargoes\cite{unsc09j}. This was immediately followed by a statement by the DPRK in which they outlined responses to the resolution, including increased attempts at uranium enrichment\cite{nti15}.

\subsection{Cycle 8}

In March of 2010, an ROK patrol ship sank near the Korean maritime border\cite{branigan}. Though the ROK initially refused to speculate on whether the DPRK was involved with the sinking\cite{branigan}, the South’s government refused to negotiate with the North until the incident could be investigated\cite{davenport}. Analysis pointing to a deliberate torpedoing by the DPRK quickly emerged\cite{reuters}, and the ROK formally announced that it would sever most economic ties with the North as a result. The next day, the DPRK announced that it would “cut all links” to the South in response to the accusations\cite{davenport}. The United States imposed additional sanctions on the DPRK, and held a joint military exercise with the ROK “to demonstrate the alliance’s resolve” and “send a strong message to Pyongyang”\cite{starr}.

\subsection{Cycle 9}

In November, barely half a year after the Cheonan torpedo incident, the DPRK shelled Yeonpyeong Island and the ROK returned fire\cite{bbc}. China called for an immediate return to the six-party talks\cite{bbc}, but several other participant states rejected on the grounds that relations between the DPRK and the ROK were not good enough for that to be reasonable\cite{davenport}.

In early 2011, the DPRK informed a Russian official that it would consider resuming the six-party talks, but the ROK rejected this offer on the grounds that there was no reason to believe in the sincerity of the DPRK’s negotiating efforts\cite{davenport}. In May, the ROK offered the DPRK a position at the Nuclear Security Summit the following year, if they would commit to denuclearization – however, the DPRK denounced this as a ruse attempting to soften the North up for invasion\cite{davenport}. Over the course of the summer, the tone of diplomacy became markedly more positive, and resumption of the six-party talks appeared more and more likely\cite{davenport}.

\subsection{Cycle 10}

In March of 2012, the DPRK announced that it would launch a satellite the following month to commemorate the centennial of Kim Il Sung – which, according to the United States, would violate the terms of the Leap-Day Agreement\cite{davenport}. Shortly thereafter, the US temporarily suspended its delivery of food aid, which was then halted completely after the satellite launch was (yet again, unsuccessfully) attempted in April\cite{davenport}.

Although the UNSC quickly condemned the launch as a violation of numerous previous resolutions regarding ballistic missile launches\cite{unsc12}, the DPRK announced its intention to try again with a similar configuration later that year. In mid-December, the launch was attempted and, for the first time, external sources confirmed that the satellite achieved orbit\cite{davenport}. In response, the UNSC passed another resolution reaffirming previous sanctions and condemnations and demanding that North Korea “abandon all nuclear weapons and nuclear programmes completely, verifiably, and irreversibly.” \cite{unsc13}

In response, the DPRK announced that it intended to continue with missile testing, and additionally that it would soon conduct another nuclear test\cite{davenport}. Seismic activity consistent with underground nuclear detonation was detected in North Korea in early February of 2013\cite{davenport}. The UNSC responded with another resolution “strengthening and expanding the scope of” existing sanctions\cite{unsc13m}.

\subsection{Summary of cycles}
\subsubsection{Beginnings and Endings}
\begin{tabular}{|l|p{6cm}|p{6cm}|}
	\hline
	Cycle (year) & Initiating Incident & End result \\ 
	\hline
	1 (1992-3) & Diplomatic engagement (DPRK files IAEA report) & US-DPRK joint statement - DPRK gains LWR tech assurances, promises to engage with IAEA \\ 
	\hline
	2 (1994) & Diplomatic engagement (DPRK allows IAEA inspectors again) & US-DPRK Agreed Framework – DPRK gains LWR tech assurances, energy aid; freezes graphite reactor programs \\ 
	\hline
	3 (1998-9) & Missile launch (DPRK attempts to place satellite into orbit) & US-DPRK agreement – temporary ban on long-range missile tests in exchange for sanction relief \\ 
	\hline
	4 (2002-3) & US action (official claims DPRK has a secret enrichment program) & Inconclusive negotiations; DPRK admits it has nuclear weapons \\ 
	\hline
	5 (2005-6) & US action (freezing DPRK assets in a Macau bank) & Major impasse to negotiations created; DPRK loses access to \$25 million in funds \\ 
	\hline
	6 (2006-8) & Missile launch (DPRK tests several missiles of different types) & 6-party agreement – rollback of reactor program, removal of DPRK from State Sponsors of Terror list, unfreezing of Banco Delta Asia funds \\ 
	\hline
	7 (2009) & Missile launch (DPRK attempts to put satellite into orbit) & Increase in sanctions, tentative but inconclusive negotiations \\ 
	\hline
	8 (2010) & Torpedo attack by DPRK & US sanctions; US-ROK joint military exercises \\ 
	\hline
	9 (2010-11) & Artillery shelling by DPRK & Inconclusive but positive-leaning negotiations \\ 
	\hline
	10 (2012-13) & Missile launch (DPRK attempts to put satellite into orbit) & Increase in sanctions, DPRK nuclear testing \\
	\hline
	\end{tabular}

\subsubsection{Notable Events} 
\begin{tabular}{|c|p{2.4cm}|p{1.6cm}|p{8cm}|}
	\hline
	Cycle (year) & Nuclear/Missile tests & Sanctions & Changes to status quo \\ 
	\hline
	1 (1992-3) & None & None & DPRK gains assurances of LWR assistance in exchange for returning to previous agreements \\ 
	\hline
	2 (1994) & None & IAEA Board of Governors & DPRK gains assurances of LWR assistance and energy aid in exchange for returning to previous agreements \\ 
	\hline
	3 (1998-9) & Long-range missile & None & DPRK gains reduction in sanctions in exchange for halting already-sporadic tests for a short but unspecified time \\ 
	\hline
	4 (2002-3) & None & None & DPRK loses plausible deniability and US energy aid \\ 
	\hline
	5 (2005-6) & None & United States & DPRK loses access to funds, gains ability to stonewall on negotiations until issue is resolved \\ 
	\hline
	6 (2006-8) & Both & Many - US, UN, other nations & Many sanctions imposed, others removed; DPRK loses access to reactor program, gains some aid \\ 
	\hline
	7 (2009) & Both & UN & Sanctions/arms embargoes imposed; DPRK regains access to reactor; ROK joins PSI \\ 
	\hline
	8 (2010) & None & US, ROK & Severance of trade/relations between DPRK and ROK; increase in US sanctions \\ 
	\hline
	9 (2010-11) & None & None & No major change – some increase in discussion about resuming six-party talks \\ 
	\hline
	10 (2012-13) & Both & UN & Multiple rounds of UN sanctions imposed \\
	\hline
\end{tabular} 

\subsection{Trends}

There are a number of variables that can be examined across the cycles of escalation outlined above. For the purposes of this paper, the two that will be examined are, first, the instigating party, and second, the presence or absence of weapons testing. A number of other variables – the presence of sanctions, IAEA involvement, timing with respect to military exercises, and more – could also provide interesting results, but limiting analysis to two factors will allow a manageable scope, which will hopefully yield more constructive results.

\subsubsection{Cycles instigated by North Korea}
There are two cycles (1 and 2) which began with the DPRK making diplomatic overtures to some external body. In both of these cases, the cycle concluded on a positive note – cycle 1 produced the US-DPRK joint statement in which the DPRK promised to engage with the IAEA in exchange for LWR construction aid, while cycle 2 followed this up with more concrete assurances in the same direction along with energy aid, in exchange for a full freeze on the North Korean graphite reactors.

One possible explanation for this phenomenon is that, though there have been many negotiations begun by the DPRK that were not in good faith, those negotiations fizzle out rather than leading to escalatory cycles. When there is escalation, parties get more invested in seeing the cycle through to some kind of significant conclusion. If escalation was begun by diplomatic overtures by North Korea, this was because it was in a political situation in which it seemed advantageous for negotiations to occur, and therefore DPRK officials were more willing to come to the table with genuine intentions.

This conclusion is limited by the fact that both cycles in which this occurred were quite early in the history of DPRK nuclear negotiations, which may mean that the international community had not yet become skeptical of North Korea’s willingness to make good on international agreements. Thus, agreements were signed – and then, many years down the line and several cycles later, broken. The theory thus cannot be confirmed or refuted until another escalatory cycle occurs in the modern political environment that begins with diplomatic overture on the part of the DPRK.

\subsubsection{Cycles instigated by the United States}
The two cycles (4 and 5) that began with US action were largely unsuccessful. In cycle 4, a US official accused the DPRK of having a secret enrichment facility, which started a series of events in which the DPRK fully withdrew from the NPT and restarted their mothballed reactors. During this process, while the DPRK regained a functioning nuclear reactor, they also saw a freeze in aid from the US, and the withdrawal from the NPT set the stage for a great deal of disagreement in years to come. In cycle 5, the US froze North Korea’s Banco Delta Asia assets, which became a sticking point for negotiations many years into the future and bogged down progress significantly.

When a cycle begins with external action, the DPRK seems to double down on its course of action and refuse to change its behavior. There is a straightforward interpretation for this behavior: both initiating events studied were adversarial in nature, and in that case it stands to reason that the DPRK would not want to be seen as caving to the aggressions of foreign powers – the United States in particular. This conclusion seems ominous for those within the US who wish to substantively influence the DPRK’s nuclear policy (and not by increasing its dedication to pursue bigger and better weapons) – but, since both examples were hostile overtures, conclusions cannot actually be drawn about externally-initiated interactions in the general case.

\subsubsection{Effects of weapons tests}
North Korea has tested nuclear weapons three times. Each time, it was in the midst of an existing escalatory cycle (cycles 6, 7, and 10), immediately following a UNSC resolution or official condemnation of a missile test. Of those, cycles 7 and 10 began with satellite launches widely decried as long-range missile tests (as did cycle 3, during which no nuclear testing occurred), while cycle 6 began with a broad range of missile tests conducted under no particular pretenses.

Cycle 3, which began with an unsuccessful satellite launch using long-range missile technology, involved a year of negotiating which concluded with sanction relief for the DPRK in exchange for a voluntary moratorium on long-range testing. This, it seems, was a net win for the DPRK, which received economic benefit in exchange for an agreement that required no action on their part.

Cycle 6 also concluded fairly positively – though it began with a missile test and involved nuclear testing and significant UNSC sanctions, the DPRK also ended the cycle with the removal of a significant freeze on funds in Banco Delta Asia (and thus, resolution of a major diplomatic roadblock), removal from the US State Sponsors of Terror list, and the relaxing of several trade barriers set up by the US government. Additionally, the DPRK agreed to allow the IAEA back into their nuclear facilities, which meant that other nations also had reason to be pleased with the final outcome of the cycle.

Subsequent cycles in which nuclear tests occurred were far less cordial in resolution. In cycle 7, a number of sanctions were imposed by the UNSC, and the political situation on the Korean peninsula ended up dramatically tenser than it was at the start of the cycle. In cycle 10, there were several rounds of sanctions, and UNSC statements that for the first time called for the complete denuclearization of the DPRK rather than simply a return to IAEA inspection regimes.

There are several likely factors at play here: first, that the missile and nuclear testing has grown more advanced – and therefore more alarming to the international community – over time. When a single, unsuccessful rocket launch occurred, it was reasonable to believe that the DPRK could be dissuaded from its military objectives. Even when the first nuclear test occurred, this was still plausible – but as the DPRK’s missile and nuclear technology became more robust (evidenced by larger nuclear detonations and a successful satellite launch), the international community continued to demand the same return to no nuclear weapons programs at all. The urgency of this demand increased with the level of technology demonstrated, but as their technology improved, the DPRK had more to lose by agreeing to reverse advancements. These opposing incentives therefore have increased the tension inherent to negotiations, and decreased the likelihood of mutually favorable outcomes.

A second relevant factor is likely that, because there have been so many cycles involving missile and nuclear escalation and the DPRK continues to have missile technology and nuclear weapons, the international community has become unwilling to believe in the sincerity of DPRK negotiations. Early on, it was possible to believe that North Korea really would roll back its programs if given the correct incentives. However, as time passed and many cycles occurred without successfully sustaining missile moratoriums and the like, people became more skeptical of this premise. Nations have begun to require more stringent and concrete assurances that the DPRK is serious, which the DPRK resents as condescension or conspiracy.

\subsubsection{The memorylessness of North Korea}
The most general statement that can be made about the behavior evident across cycles is that the DPRK acts as though it has no memory. That is, in each new situation, it reacts with outrage to international censure, protesting that it has always negotiated in good faith and is being persecuted by the unjust and greedy actions of the United States or South Korea. In short, it acts in a way that would be perfectly reasonable, had military exercises or sanctions occurred in a vacuum – however, when these responses come after actions that are themselves responses to aggressive action like missile tests, this position begins to appear less justifiable.

The first possible interpretation for this behavior – and also the least interesting from the standpoint of critical analysis – is that North Korea is simply irrational, and honestly maintains the delusion that its aggressive actions should have no consequences. This is not a terribly compelling argument – the state has, after all, maintained autonomy and international importance disproportionate to its size and economic clout for a great many years, which seems implausible if it is simply acting on whims.

Another interpretation is that the DPRK believes it is important to maintain an image of persecution – which could be due to a range of internal and external factors. It is possible, for example, that they intend to use this rhetoric to bolster the legitimacy of their cause in the eyes of nations that are not directly involved, but who may be valuable allies – even if the rhetoric is not borne out by the facts, nations unsympathetic to the US and their allies may latch onto it as justification for alliance with the DPRK. Additionally, it is possible that they intend this rhetoric primarily for a North Korean audience – given tight controls on non-official news sources, portraying the DPRK as persecuted but defiant may be highly advantageous in terms of maintaining patriotic sentiment and support for any actions taken within escalatory cycles.

\subsection{Conclusions}
In early cycles, the DPRK tended to come out of the cycle having gained either direct material benefit (food, oil) or assurances of future benefits from other countries. In exchange, they made agreements that, for the most part, would have them roll back their nuclear program to the state that it was in prior to the beginning of the escalatory cycle.

However, as time passed, the international community became less willing to accommodate the DPRK’s behavior – perhaps because they learned from previous experiences in which the DPRK did not follow through on its commitments, or because the DPRK’s actual behavior exceeded some threshold of unacceptable provocation. Regardless of cause, negotiations with the DPRK in recent years have involved more stringent demands and strict preconditions created by the US and other nations seeking to avoid the cyclic behavior exhibited in the past.

%The trends illustrated by analysis of the escalatory cycles have alarming implications for American foreign policy. Cycles in which the US initiated diplomatic actions resulted in markedly worse outcomes than cycles with DPRK-initiated diplomacy and even some cycles kicked off by missile tests. Since both cycles initiated by the United States began with hostile actions, it cannot quite be concluded that the US can never expect useful outcomes if it initiates an escalatory cycle. However, it is certainly a sobering indicator that the US cannot expect heavy-handed demands to produce useful results.

%This is particularly awkward in light of the shift in the tone of negotiations over time – it seems that ultimatums are currently the policy option du jour. For over a year, both South Korea and the United States have refused to negotiate with North Korea until it agrees to complete denuclearization\cite{yohnap}, and, unsurprisingly, the DPRK has not responded positively\cite{kcna2}. Rather than a sign that the correct combination of sanctions and other pressures has not yet been found, perhaps American diplomats should take this as an opportunity to tone down their rhetoric – and maybe avert the doubling-down that has been seen in past cycles.

The historic record seems to show that negotiating with North Korea is most successfully done when an escalatory cycle begins with DPRK overtures. This is unsurprising – the DPRK comes to the table intentionally, without apparent coercion to justify unproductive posturing. Uncomfortable as it is for American negotiators to admit, it may well be the case that the best opportunities to get a serious foot in the door of the North Korean nuclear agenda come at the behest of the North Korean government. Rather than the timeline that best suits the United States, negotiators may be best served by waiting for opportunities and seizing them as they come.

\todo{needs 2016 update!!}
\todo{maybe move conclusions to end}
\section{Technical Analysis}

%I feel like actually this wants to go first or last - tell us about what north korea can currently do, use it to either frame the history or provide context for what's currently/will eventually occur

\subsection{Methodology}

A complete picture of the North Korean bargaining strategy cannot be formed without examining the technical data that exist on the DPRK's nuclear program. This general category encapsulates several subsections: production capacity, material stockpiles, and missile technology.

Production capacity analysis will draw initially from the technical details reported by analysts like Hecker\cite{hecker} who have observed North Korean enrichment and reprocessing facilities firsthand. Further analysis will draw preferentially from information provided by NGOs, with gaps filled in by information released by American and DPRK sources. (It is believed that government estimates are more likely to be biased in one direction or the other, so these sources will be avoided unless no other estimates exist)

Stockpile estimation will examine the amount of nuclear material shipped into the DPRK as part of various treaties and agreements over the years, as well as any estimates that exist on material that may have been smuggled into or out of the country. Production capacity analysis will be used to suggest how much bomb-grade material the DPRK has now or could straightforwardly produce based on initial materials estimates.

Examination of missile and bomb technology will include both weapon miniaturization (the process by which nuclear warheads can be made small enough to fit on a missile) and the status of the DPRK's missile program. The outcomes of recent missile and nuclear tests will be examined, and expert analysis on the magnitude of explosions, level of advancement of rockets, etc. will be considered. 

In conjunction, this information reveals a great deal about North Korea's position in nuclear negotiations. "Breakout time", or the amount of time necessary to produce a bomb's worth of weapons-grade nuclear material, was a critical discussion point in Iranian nuclear negotiations. It seems likely, therefore, that the speed with which the DPRK could credibly produce a weapon might impact how the US negotiates with them - the easier it is for North Korea to make a bomb, the more difficult the process of convincing them to give up the technology may become - and the more urgent the problem becomes. Additionally, as the amount of work that stands between the DPRK and the ability to threaten nuclear strikes on distant capitals decreases, a similar calculus occurs.

\todo{'technical analysis in its entirety'}

\section{Rhetorical Analysis}

\subsection{Methodology}
The development of the escalatory cycles model in the historical section forms the starting point of rhetorical analysis. As the relationship between the DPRK and the US shifts through rising and falling tensions, it is expected that the rhetoric used by official outlets on either side will shift commensurately.

This rhetorical examination will focus on the evolution of diplomatic language over time - for example, the phrases used by official North Korean news to describe political adversaries, or the terms US officials emphasize when discussing previous agreements. These are valuable indicators of general attitude, and when taken in conjunction with the outcomes of the associated negotiating efforts, can suggest correlations between various rhetorical tendencies and the prospects for successful negotiation.

This work will draw upon previous efforts by Rich \cite{rich12, rich14} and Sin \cite{sin}, who each used data drawn from a decade of articles from the English-language website for the KCNA (North Korea's ruling party's official news source intended for foreign eyes) to track trends in the DPRK's nuclear rhetoric. The KCNA website is an extremely useful source of information on the views and policies of the North Korean government. It is produced directly by the government for foreign consumption, which means that patterns found within its rhetoric reveal (intentional or accidental) signaling of agenda by the North Korean government rather than the biases of foreign translators \cite{rich12}. Trends discovered within these data will be compared across escalatory cycles in order to understand the way in which North Korean rhetoric tracks with the state of nuclear diplomacy.

No automated content analysis exists for similar statements from the United States, and producing such analysis is too complex an undertaking for this thesis. However, this does not preclude rhetorical analysis of the American side of the picture. Individual examination of key speeches and statements during the various escalatory cycles will produce less robust analysis than would be possible using large-scale data-scraping and analytics, but may still provide useful indicators of changes in American attitudes.

\todo{write more about American language}

%basically what we're gonna do here is take Rich and Sin's KCNA analysis as a starting point for that side of rhetoric and probably not do much independent analysis. meanwhile, while bleiker is super useful for initial rhetoric analysis of the US, we will focus our independent examinations on that angle since it's less well-established. Maybe we'll do some examination of KCNA docs around key times ourselves, but yeah.

\subsection{General Rhetorical Trends}
\subsubsection{North Korea}

Rich \cite{rich14} examined KCNA articles from 1997-2012 in order to develop a profile of topics and nations most frequently mentioned in conjunction with nuclear issues. In examining member nations to the six-party talks, he discovered something interesting -  the United States was near the bottom in terms of reference frequency, but was more strongly correlated with nuclear mentions than any other nation. Since the six-party talks have been the primary locus of nuclear diplomacy during the studied timeframe, it is expected that the countries party to the talks would be mentioned in a nuclear context more than other countries.

The fact that the United States is disproportionately referenced in articles discussing nuclear matters is confirmed by additional research conducted by Rich on the year of the Yeonpyeong and Cheonan incidents, during which "increase in references to the United States, holding all else constant, [was] the largest predictor of additional references to nuclear issues" - even though in that year, references to the United States altogether were notably rarer than any other six-party talks member and barely above references to Mexico \cite{rich12}.

This suggests that, though the nuclear diplomacy effort with the DPRK is intended to be a multilateral affair, North Korea nevertheless considers the United States to be the most relevant foreign party to the conversation \cite{rich14}. This is unsurprising, as they have explicitly referenced American nuclear proliferation as the motivator and justification for their own nuclear agenda \cite{kcna, kcna3, kcna4}.

Confirming DPRK focus on the United States suggests that analyzing the nuclear issue from a primarily bilateral standpoint is in fact valid - shifting messages and the like are clearly crafted with American eyes in mind. But more generally, this provides evidence that American preference for developing broad multilateral support for negotiations is misguided - if the DPRK cares mostly about what the US thinks, adding more voices from other countries likely provides at best marginal gains.

\subsubsection{United States}
\emph{Bleiker's notes: 'rogue state' language, administration-specific but also overall}

\emph{anything of note that appears later and seems to belong here}

\todo{atemporal US rhetoric trends}

\subsection{Cycle-based Analysis}
Though some general trends can be identified in American and North Korean rhetoric largely independent of the passage of time, more interesting for the purposes of this examination are the links between escalatory cycles and changes in the language of diplomacy. 

\subsubsection{DPRK}

rich14 initially says "rhetoric ramps up before nuclear tests, then draws down dramatically afterwards", while rich12 says "rhetoric draws down dramatically during yeonpyong and cheonan"

Sin says single largest of their categories before and after incidents is nationalism, but that's also the largest overall so it's unclear if it's differentially relevant - we're going to have to reframe

-rhetoric does not seem to change correlated to liberal/conservative presidential administrations in either US or SK, despite increases and decreases in hostile rhetoric ('rogue state', 'axis of evil') as per bleiker

-this is harder to confirm though, because references increase over time (rich sez) which could be either due to accumulating hostilities or just general increase in focus on nuclear issues (I think)

(general for this section: Whatever happens when we reframe rich14, what we see from rich12, whatever falls out of Sin)


\todo{dprk temporal rhetoric}


\subsubsection{US}
bleiker reframed

cycle-by-cycle examination of US press statements re: stuff

\todo{cyclic US rhetoric}

\subsection{Cycles summarized}
probably Yet Another Table
\todo{rhetoric table}

\subsection{Cyclic Trends Identified}
Cycle correlations
\todo{rhetoric conclusions}

\section{Results and Discussion}
\todo{results}

\section{Conclusion}
\todo{conclusions}

\bibliographystyle{ieeetr}
\bibliography{references}

\todos

\end{document}
