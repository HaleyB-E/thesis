\documentclass{article}
\usepackage[utf8]{inputenc}

\title{Thesis Introduction and Background}
\author{Haley Brandt-Erichsen}
\date{November 2015}

\begin{document}

\begin{titlepage}
    {\centering
    Rethinking American Negotiation Strategies in North Korean Nuclear Talks \par
    \vspace{0.5cm}
    by \par
    \vspace{0.5cm}
    Haley Brandt-Erichsen \par
    \vspace{0.5cm}
    Submitted to the \par Department of Nuclear Science and Engineering \par in Partial Fulfillment of the Requirements for the Degree of \par
    \vspace{0.5cm}
    Bachelor of Science \par
    \vspace{0.5cm}
    at the \par
    \vspace{0.5cm}
    Massachusetts Institute of Technology \par
    \vspace{0.5cm}
    June 2016 \par
    \vspace{0.5cm}
    \copyright 2016 Haley Brandt-Erichsen \par All rights reserved \par
    \vspace{0.5cm}
    The author hereby grants to MIT permission to reproduce and to
distribute publicly paper and electronic copies of this thesis document in whole or in part
in any medium now known or hereafter created. \par
    }
    \vfill
    
    {Signature of Author: \_\_\_\_\_\_\_\_\_\_\_\_\_\_\_\_\_\_\_\_\_\_\_\_\_\_\_\_\_\_\_\_\_\_\_\_\_\_\_\_\_\_\_\_\_\_\_\_\_\_\_\_\_\_\_\_\_\_\_\_\_\_\_\_\_\_}
    \begin{flushright}
        Department of Nuclear Science and Engineering \par 
        DATE GOES HERE
    \end{flushright}
    \vspace{0.5 cm}
    {Certified by: \_\_\_\_\_\_\_\_\_\_\_\_\_\_\_\_\_\_\_\_\_\_\_\_\_\_\_\_\_\_\_\_\_\_\_\_\_\_\_\_\_\_\_\_\_\_\_\_\_\_\_\_\_\_\_\_\_\_\_\_\_\_\_\_\_\_\_\_\_\_\_\_\_\_\_\_\_\_\_}
    \begin{flushright}
        R. Scott Kemp \par
        Assistant Professor of Nuclear Science and Engineering \par
        Thesis Supervisor
    \end{flushright}
    \vspace{0.5 cm}
    {Accepted by: \_\_\_\_\_\_\_\_\_\_\_\_\_\_\_\_\_\_\_\_\_\_\_\_\_\_\_\_\_\_\_\_\_\_\_\_\_\_\_\_\_\_\_\_\_\_\_\_\_\_\_\_\_\_\_\_\_\_\_\_\_\_\_\_\_\_\_\_\_\_\_\_\_\_\_\_\_\_\_}
    \begin{flushright}
        Michael P. Short \par
        Assistant Professor of Nuclear Science and Engineering \par
        Chairman, NSE Committee for Undergraduate Students
    \end{flushright}
    
    
\end{titlepage}

\newpage
\thispagestyle{empty}
       \addtocounter{page}{-1}
       \null
\newpage

\tableofcontents

\pagebreak

\section{Acronyms Used}

DPRK – Democratic People’s Republic of Korea (North Korea)

\noindent IAEA – International Atomic Energy Agency

\noindent KCNA – Korean Central News Agency (North Korean state-owned news source)

\noindent KEDO – Korean Peninsula Energy Development Organization

\noindent LWR – light-water reactor

\noindent NPT - Treaty on the Non-Proliferation of Nuclear Weapons (Non-Proliferation Treaty)

\noindent ROK – Republic of Korea (South Korea)

\noindent UNSC – United Nations Security Council

\section{Introduction}

\subsection{Motivation}

The specter of North Korean nuclear weapons has haunted the world since the early days of their nuclear reactor program in the 1960s \cite{pincus}. The international community, in particular the United States and South Korea, has always been adamant that a nuclear-armed North Korea is intolerable \cite{kerry,parksk}.

In the past, many believed that it was possible to achieve peaceful nuclear power for the DPRK without undue risk of weapon production – indeed, the 1994 Agreed Framework outlined a plan in which the United States would help North Korea develop a light-water reactor program \cite{agreed}. However, in recent years, even “proliferation-resistant” reactors like the LWRs have been deemed too great of a risk. In 2008, Lee Myung-bak ran his presidential campaign in South Korea on the founding principle that aid to the North should be dependent upon complete denuclearization \cite{snyder}. As of late 2015, the United States held the official position that resuming multilateral negotiations with the DPRK could not occur until denuclearization was guaranteed \cite{pennington}. North Korea, meanwhile, is insistent that it has the right to nuclear materials for both power plants and weaponry \cite{kcna}. 

A variety of methods have been tried to resolve this disagreement, from sanctions and aid restrictions to mandated denuclearization as a condition on resumption of treaty talks \cite{bajoria,davenport}. These methods have, at best, resulted in temporary disarmament or a partial rollback of the DPRK’s nuclear programs – but, in every case, the changes were only temporary and North Korea eventually returned to developing its nuclear weapons technology \cite{davenport,nti15,iaea09}.

When these methods have not succeeded, the UN and individual countries attempting to change the DPRK’s behavior have not systematically altered their approach \cite{davenport,nti15}. This indicates that negotiators believe the strategies themselves are valid, and grounded in a correct understanding of the DPRK’s nuclear situation. It is worthwhile to investigate whether this assumption is warranted.

\subsection{Objectives}

This thesis will combine analysis along three axes in order to practically judge North Korea's bargaining position. Examination of the historic record of negotiation proceedings will be combined with analysis of the rhetoric of press releases and official government statements, as well as calculations on nuclear material stockpiles, weapon testing yields, and production capacity. Unlike previous research, this combines qualitative and quantitative analysis in order to form a more complete picture of the DPRK’s objectives and abilities.

It is hypothesized that United States strategy in negotiating with North Korea is inconsistent with the technical facts of the North Korean nuclear program and past negotiating behavior. In other words, the cyclical nature of negotiations persists because negotiators base their positions on what they want to be true about the DPRK, rather than the facts of the situation. 

\section{Background}

\subsection{Brief history of North Korean nuclear programs and negotiation attempts}

Throughout the 1950s, the DPRK and the USSR worked together to initiate the North Korean nuclear power program, beginning construction on the Yongbyon nuclear complex in the early 1960s \cite{nti15}. The country became party to limited IAEA regulations in the following decade for the 5 MWe reactor constructed at Yongbyon, but did not develop a complete safeguards agreement until some time after its entrance into the NPT in 1985 \cite{iaea09}. During the 80s, North Korea continued to expand the Yongbyon complex and look into LWR technology \cite{ntiYongbyon}.

Over the next several decades, the US and the DPRK signed a number of agreements under which assurances were made about the provision of LWRs to North Korea in exchange for regulations intended to limit risk of proliferation \cite{agreed, davenport}. However, KEDO – the organization created to implement these agreements by constructing the LWRs – faced significant challenges in terms of funding and political support, eventually functionally collapsing under the strain of disagreements between the DPRK and member states in 2006 \cite{kedohist}.

Simultaneously, the Yongbyon reactors that the KEDO project was intended to replace existed in a constant state of flux. The complex would be shut down pursuant to some new agreement, or the IAEA would be allowed to inspect portions of the site…but then, a few years later, the inspectors would be expelled and the facilities restarted again \cite{davenport,nti15,iaea09}.

As the reprocessing facility in Yongbyn operated in fits and starts, the DPRK gradually accumulated enough plutonium to produce nuclear weapons. An underground nuclear test in a DPRK facility in 2006 shocked and horrified the international community, which immediately responded with political admonishments and economic sanctions \cite{davenport}. To date, the DPRK has tested nuclear weapons twice more, along with a number of missile tests going back to 1998 \cite{orfall} and including launches in 2013 that culminated in the successful placement of a satellite into orbit \cite{davenport}. In addition to the nuclear tests, missile tests and rocket launches by the DPRK have been roundly criticized by the UN and many individual nations as attempts to develop effective nuclear missiles \cite{unApr12}.

PROBABLY MENTION THE 2016 UPDATE HERE...

\subsection{Current Political Status}

The North Korean nuclear program has been shut down and restarted a number of times, pursuant to various agreements made and then broken between the DPRK and a number of negotiating bodies \cite{bajoria,davenport}. The DPRK insists that its peaceful program should be allowed to continue \cite{kcna2}, while the US and South Korea claim that even this is unacceptable \cite{lee} – and demand complete commitment to denuclearization before any further negotiations can even begin.

This hard-line position is informed by fact – the Yongbyon reactor, nominally a peaceful source of power, did produce the material that has been used in the DPRK’s weapons program \cite{hecker}. But the stance has proven unhelpful in getting North Korea to the table for negotiations – they claim that it is their right as a sovereign nation to pursue both peaceful and military nuclear technology, and that maintaining this right is a necessary deterrent against foreign aggression \cite{kcna,kcna2}.

Each side, then, is demanding as a precondition for negotiation the very action that the other side refuses to even consider. It is perhaps unsurprising that this has not been effective. The result has been a cycle of behavior that has gone on for several decades: aggression and posturing on North Korea’s part is followed by sanctions and demands by other nations. Tension escalates until one side or the other begins to call for a diplomatic solution, which either ends in stalemate or produces results that only last for a short time before the agreements are broken again \cite{bajoria, davenport}.

\subsection{Previous Analysis of the Problem}

Many analyses of the nature of the politics surrounding DPRK nuclear diplomacy exist, and several have concluded that the problem is cyclic in nature \cite{blair,cfr,fisher,gause,habib,jun}. In general, the issue tends to be modeled as a series of cycles comprised of initial posturing, followed by aggression and escalation, and concluding with reconciliation that, depending on the analyst, may or may not be identified as sincere.

When these analyses are politically-motivated \cite{blair, cfr}, their logic tends to begin with the initial proposition that demand for complete denuclearization can someday be fulfilled, if only the correct combination of diplomacy and coercion can be found that will force the DPRK to comply. More academic treatments of the problem \cite{habib,jun} vary widely in their approach. Jun \cite{jun} identifies problems with past approaches, but seems to conclude that increased coordination between negotiating bodies and oversight of agreement implementation may yet salvage the denuclearization agenda. Meanwhile, Habib \cite{habib} claims that North Korea’s cyclic behavior is inevitable due to the nation’s military-first ethos. He further postulates that the nuclear program is a key part of this ethos, and thus that the DPRK will never sincerely agree to give it up.

Some rhetorical analysis of the North Korean nuclear issue does exist - in particular, one author has used automated frequency analysis to look at how references to countries and specific incidents correlates with references to nuclear programs in the DPRK’s official news agency's articles \cite{rich12,rich14}.

Through analysis of key words used in 15 years of KCNA articles, Rich uncovered a strong link between mentions of the United States and the DPRK’s nuclear program \cite{rich14}. Additionally, during a time period marked by international concern over North Korean military posturing and escalation, mentions of their nuclear program in KCNA articles were few and far between \cite{rich12}.

While this information does allow valuable conclusions to be drawn about the DPRK’s political positions, analysis of the \emph{tone} of KCNA articles and how this relates to the success or failure of negotiations they report on does not appear to exist. Neither, it seems, does any substantial analysis of United States rhetoric in press releases related to North Korean nuclear issues.

THINGS THAT STILL NEED TO HAPPEN HERE:
-expansion on cyclic analysis - is there underfitting? (most seem to postulate all escalatory cycles follow a single form, and this seems incorrect
-what is the impact of rich's frequency analysis (probably actually read rich to get this)
-is there any tonal analysis of KCNA or anything else
-what previous mathematical info exists that isn't classified?

\section{Methodology}

\section{Historical Analysis}

\section{Technical Analysis}

\section{Rhetorical Analysis}

\section{Results and Discussion}

\section{Conclusion}


\bibliographystyle{ieeetr}
\bibliography{references}

\end{document}
