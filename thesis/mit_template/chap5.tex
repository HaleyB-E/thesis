\chapter{Diplomatic Evolution}

In order to understand the current negotiating positions of both the United States and North Korea, it is instructive to examine the history of the diplomatic postures of both countries over the past several decades. The historical and rhetorical analyses above each form an important component of the larger picture. Rhetorical analysis showed how the countries talked about what they did, while historical analysis identified the events triggering and being triggered by national positions. To tie these together, the evolution of positions themselves must be examined.

This section will use political administrations as breakpoints for analyzing diplomatic changes over time. The changes in priorities and negotiating strategies across different American presidential administrations are obvious. Shifts due to changes in North Korean leadership may be more difficult to identify, but there are still some notable patterns.

\section{Pre-1994}

The official American policy on North Korea under George H.W. Bush was one of ``strong engagement'', under which the DPRK would be allowed minimal access to nuclear material and steered heavily towards noronproliferation \cite{cerami}. This, in practice, translated to ``coercive diplomacy'' that relied far more heavily on coercion than diplomacy. Despite occasional forward steps, the administration did not seriously believe progress was possible through negotiation \cite{sigal}.

When Clinton took office, neither the official policy nor its implementation initially changed. Indeed, resumption of joint US-ROK war games (a policy decision made by the departing Bush) paired with dramatic rhetoric about the threat posed by North Korea followed quickly on the heels of Clinton's inauguration \cite{cumings}. During cycle 2, North Korean concerns about American meddling with the IAEA's inspection objectives combined with dismay over the military exercises to produce a climate of brinksmanship and hostility \cite{cumings}.

\section{Clinton and Kim Jong-il, 1994--2000}

\subsection{Development of the Agreed Framework}

From the American perspective, the DPRK was trying to evade its responsibilities pursuant to the IAEA. However, the DPRK felt that the IAEA's demands were unreasonably stringent, and worse, shaped by an American agenda to punish North Korea for insufficient deference \cite{cumings}. The DPRK withdrew from the IAEA and refused to allow inspectors into its nuclear facilities \cite{iaea09}.

Meanwhile, the death of Kim Il-sung meant that Kim Jong-il took power in the midst of a crisis, facing American threats of both economic sanctions and military power \cite{jun,cerami,cnn99}. The latter was particularly vexing to the new leader, for whom cancellation of war games the previous year had been a point of personal pride \cite{farrell}. The former also specifically coincided with a time of economic weakness and stagnation in the DPRK. The late days of Kim Il-sung's regime saw a slight but important opening of the North Korean economy to some foreign trade \cite{sigal}, which brought with it new avenues by which international pressure could be brought to bear.

While the DPRK would likely have come out on the bottom of a military engagement, this did not mean that war would serve American interests. The Yongbyon site could not be targeted without risking serious regional radiation release \cite{beal}. Even absent that concern, definitively crippling the North Korean nuclear program would require more confidence than US officials possessed that there were no nuclear sites that remained secret \cite{sigal}. Total war with North Korea would cost many thousands of lives---American, and North and South Korean---and was thus similarly unpalatable \cite{cumings}. Thus, though the military threat posed by the United States to the DPRK was serious (and came extremely close to realization \cite{cnn99}), it was not something American politicians wanted to do given any alternative.
	
The stage was set, therefore, for productive negotiation between the two parties. Both felt that they had been wronged by the way the other was interacting with the IAEA. The DPRK was backed into a corner, with a relatively untested leader facing serious threats to his nation that could no longer be averted by stonewalling. And the United States, also under new leadership, found the harsher methods in its diplomatic arsenal too risky \cite{hecker2}.

Though international pressure played a role in the crisis, the actual acts of diplomatic engagement were bilateral. This fits with empirical data suggesting that the DPRK's nuclear posture has always been primarily shaped by US-DPRK politics \cite{rich14}. The crisis was defused by the development of the Agreed Framework, negotiated with the aid of former president Jimmy Carter. This agreement, under which the DPRK froze its nuclear program in exchange for concrete guarantees of energy assistance \cite{agreed}, was very much in line with the goals of the first Bush administration's North Korea policy, which the Clinton administration made a point of continuing \cite{cerami} - although the methods of engagement employed in its negotiation were far more conciliatory than had previously been utilized \cite{sigal}.

\subsection{After the Agreed Framework}

Another diplomatic crisis emerged four years after the Agreed Framework was signed. It was initiated by a DPRK-claimed satellite launch that most onlookers viewed as a missile test---a dangerous leap forward in the path towards nuclear weapons \cite{orfall}. This was a major test of the diplomatic principles at play during Agreed Framework negotiations, not least because its implementation had been hindered by those same voices in Congress who initially opposed its negotiation \cite{hecker2}. Three months later, the United States offered sanction relief in exchange for termination of the DPRK missile program, but the DPRK argued that this sanction relief should already have occurred as part of the Agreed Framework and the fact that it had not yet was a failure on the part of the Americans \cite{davenport}. This argument was presumably referencing the Framework's mandate that both nations ``reduce barriers to trade and investment'' with the other within three months of signature \cite{agreed}, which had not been followed owing to earlier disputes regarding ballistic missiles \cite{davenport}.

The negotiation of this cycle coincided with the release of the Perry Report, which notably advocated for a change in American diplomatic attitude towards a stance that favored carrots as well as sticks \cite{perry}. Though the goal was still curtailing North Korean proliferation, this marked the apex of the softening attitude towards agreements seen under Clinton \cite{bleiker}. Indeed, cycle 3 concluded with the first agreement between the two countries under which the United States acknowledged North Korea's right to exist, and was immediately followed by talks between Kim Jong-il and the US Secretary of State \cite{hecker2}.
			
The Clinton era was thus marked by a ``coercive diplomacy'' that grew ever more willing to focus on the diplomacy side---at least within the Executive. Substantive action occurred in the form of negotiations resulting in serious agreements, but pushback from conservatives in Congress limited the efficacy of their implementation \cite{harnisch}. This was, perhaps, a sign of things to come once Bush took office.
	
\section{Bush and Kim Jong-il, 2001--2008}

\subsection{Collapse of the Agreed Framework}

From the outset, the Bush administration had a very different view of US-DPRK relations. Focus shifted away from non-proliferation as a universal good, and negotiation with unfriendly nations became a sign of unacceptable weakness \cite{bleiker}. Nowhere was this better exemplified than Bush's 2002 State of the Union speech, during which he identified North Korea as a member of an ``axis of evil'' supporting global terrorism \cite{sotu02}. This was shortly followed by a number of Defense Department documents emphasizing the legitimacy of preemptive nuclear strikes and other military actions against countries seeking weapons of mass destruction \cite{bleiker,npreview}.

As the US president and Defense Department ramped up hostile rhetoric against North Korea, the State Department still clung to Clinton-era negotiation \cite{harnisch,armitage}. This split between departments of the US government did little to improve the situation, producing confusion and mistrust in both the DPRK and regional American allies.

Meanwhile, the DPRK used Bush's ``axis of evil'' comments to justify a renewed focus on nuclear weapons development \cite{bleiker}. Discussion of nuclear matters in public North Korean media increased dramatically \cite{rich14}. With the United States on the record as mistrusting North Korea to the point of considering preemptive military action, production of nuclear weapons became an imperative for survival \cite{hecker2}.

In this environment, the first official visit by a senior American official under the Bush administration turned swiftly to disaster. The DPRK viewed the American group's attitude as ``high handed and arrogant'' \cite{kcna3}, and even Western diplomats present at the time agreed that the Americans immediately began with accusations \cite{bleiker}. The result was North Korea admitting to a secret uranium enrichment program, which set off the chain reaction of events comprising cycle 4: suspension of the Agreed Framework, withdrawal from the NPT, and expulsion of IAEA inspectors \cite{iaea09}. 

The collapse of these agreements, at face value a major blow to American interests in North Korea, suited the Bush agenda. The Agreed Framework had been a thorn in the side of the Bush administration, particularly in light of its provisions with respect to an LWR program and the base fact that it constituted a substantive agreement with an ``evil'' nation \cite{harnisch}. The DPRK's withdrawal from the NPT was irrelevant to an administration more concerned with cementing the military supremacy of the US and its allies than in maintaining global non-proliferation norms \cite{huntley}.

In sum, the Bush administration's preoccupation with black-and-white diplomatic morality and their desire to position the US as the world's policeman sounded a death knell for the progress previously made in US-DPRK relations. They primed North Korea with dramatically hostile rhetoric and barely-veiled threats, then used the resultant hostility as an excuse to cancel unwanted agreements while retaining the moral high ground.

Of course, this was enabled in large part by the fact that the DPRK did, in fact, pursue nuclear weapons and admitted to doing so when pressed. North Korea was a convenient moral scapegoat for the Bush administration, but it would not have become so absent its nuclear history and apparent ambitions. There is some possibility that they did not actually possess nuclear weapons at the time of this admission, since they only claimed to have \emph{pursued} them \cite{harnisch}, at which point their choice to reveal the program might suggest they also desired to rid themselves of the agreements, which restricted their nuclear development. 

\subsection{Rise of the Six-Party Talks}

Despite the Bush administration's evident hostility towards the concept of negotiating directly with North Korea, they were still verbally insistent that they hoped for and believed in a diplomatic solution \cite{bleiker}. The compromise seemed to come in the form of a renewed focus on multilateralism with the six-party talks. This model seemed well suited to the notion that increasing coercive pressure on North Korea would force them to come to the table on American terms. And though Bush did not exactly champion global non-proliferation norms, his administration was still perfectly willing to leverage the power of international coalitions to press the DPRK. This approach, however, proved far less effective than might be desired.

First, the six involved countries (China, Japan, both Koreas, Russia, and the US) came to the table with a wide array of incentives and goals. If the US intended to show North Korea that its behavior was unacceptable, it would first have to convince nations with complicated and frequently opposing priorities to present a unified front. Given that diplomats from most other parties pointed to American intractability as a major impediment to substantive progress \cite{park6pt}, trying to build a coalition around this position was a tall order indeed.

It seems implausible that the Bush administration could not predict that its dogmatic bargaining style would mesh poorly with the intricacies of multilateral diplomacy. Nor is it surprising that this exacerbated existing tensions between the desires of other members of the six-party talks. Given this, it is not unreasonable to conclude that the six-party talks were, if not set up for failure, at least not expected to succeed. They created an appearance of progress by bringing North Korea to talks without requiring the US to soften its position, a feat that could not have been accomplished in primarily bilateral diplomacy. This insulated the US from the consequences of its attitude by providing many alternate sources of blame when talks broke down.

Though the six-party talks were designed to bring the perspectives of all six nations to bear, the DPRK's focus on bilateral interactions with the United States \cite{rich14,park6pt} contributed heavily to their successes and failures. Cycle 5 demonstrated how unilateral American action taken to limit the DPRK's economic resources ground the six-party talks to a halt \cite{greenlees}, and reversal of that action during cycle 6 almost immediately preceded the talks' first action-oriented agreements \cite{js5,js6}. 

Once again, the Bush administration's distaste for letting ``evildoers'' benefit from its policies nearly scuppered substantive diplomacy when the US failed to follow through on promises to remove the DPRK from its State Sponsors of Terror list \cite{nti15}. The US only caved once the IAEA warned that North Korean fuel reprocessing was imminent \cite{iaea09}. Given the Bush administration's prior behavior, this is hardly surprising---it would have otherwise been difficult to imagine a president so focused on moral righteousness in American foreign policy making a symbolic move like removing a sworn enemy from a list identifying them as such.

Although the Bush administration's latter days contained more dialogue and fewer diatribes than they began with, all-or-nothing demands and foot-dragging on agreements were still the standard operating procedure. During this time, North Korea made serious advancements in its nuclear program and removed itself from international agreements intended to curb its progress. They began to conceptualize of their role in negotiations as one of an established nuclear power, rather than merely a country seeking nuclear technology \cite{hecker2}. However, they did continue to come to the table for negotiations---claiming repeatedly to be willing to make changes, but not without getting something in return. That last part perhaps best summarizes the conflict during this time period---the Bush administration wanted changes, but could not accept that making changes would require them to make concessions.

\section{Obama and Kim Jong-il, 2009--2011}

American policy towards North Korea once again underwent a radical shift under Obama, at least rhetorically. In his inaugural address, he emphasized willingness to engage in negotiations even with historic enemies \cite{obama}, and during his campaign he promised to focus on engagement with the DPRK \cite{delury}. These principles were tested early in Obama's first term with the events constituting cycle 7---including both a missile launch and a nuclear test.

The motivation for these actions may have been primarily political---a loss of patience on the North Korean side for waiting for the US to blink first. It is also likely, however, that scientific opportunism played a role---with a strong desire to improve their nuclear weapon technology and an interest in minimizing potential consequences, it made sense for the DPRK to conduct this testing while the Obama administration was still gathering its thoughts on foreign strategy \cite{hecker2}.

A third potential motivating factor was the perceived strength of the Kim regime. After rumors of Kim Jong-il's ill health circulated in late 2008, a show of military power like the missile test followed by escalation to a nuclear test may have been intended to shore up the North Korean leader's reputation \cite{hecker2}. KCNA coverage of this nuclear test focused more on the international community than it did specifically the United States (unlike other nuclear and non-nuclear military incidents) \cite{sin}. This suggests that the third explanation is at least more persuasive than the notion that Kim Jong-il was testing the waters of the Obama administration. Of course, the most mundane explanation---that the test occurred early in Obama's presidency because that happened to be when North Korea was first able to do the tests it wanted---is also plausible.

After cycle 7, US policy crystallized as a strategy of ``strategic patience'' \cite{crs13}, under which it was believed that the status quo would provide sufficient incentives for North Korea to initiate diplomatic interactions absent further American pressure \cite{snyder2}. This strategy relied upon the notion that existing sanctions and agreements would slow the growth of the North Korean nuclear program to the point where no serious breakthroughs could occur. Though the State Department under Obama disavowed the notion of continuity between their North Korea strategy and that of the Bush administration, the substantive differences between the two was perhaps less than might be inferred from their rhetoric \cite{delury}. The Bush administration extricated itself from most agreements related to the DPRK and only began halfhearted negotiations again relatively late. Meanwhile, the Obama administration made little more than token efforts to maintain the level of negotiations inherited from Bush, and mostly appeared to intend to leave North Korea to its own devices \cite{delury}.

While waiting for the DPRK to make the first move, American strategy also re-emphasized coordination with regional allies---while the stated objective was an eventual return to multilateral negotiations in the form of the six-party talks, the Obama administration oversaw a much greater degree of coordination between American diplomatic posture and that of South Korea and Japan than had occurred under Bush \cite{crs13,snyder2}. This kind of coordination could serve to mitigate some of the troubles that plagued previous rounds of talks, were they ever to resume.

On the other hand, linking American and South Korean diplomatic strategies tipped the scale back from ``strategic patience'' to aggressive action during the events of cycles 8 and 9---in line with the ROK, the US imposed sanctions on North Korea after the sinking of the Cheonan, and refocused on joint military exercises in the area \cite{starr}.

North Korean news coverage did its best to segregate the Cheonan and Yeonpyeong incidents from discussion of nuclear proliferation in an apparent attempt to maintain the status quo on nuclear matters \cite{rich12}. Unfortunately for the DPRK, international media made no similar attempts \cite{beal}, and the United States responded to the incidents with sanctions and an increased interest in cooperating with the hawkish policies at the fore of South Korean rhetoric \cite{crs13}.

In the months immediately prior to Kim Jong-il's death, the US began negotiating what would eventually become the Leap Day Agreement. The policy of ``strategic patience'' had clearly failed. In late 2010, DPRK officials had revealed a uranium enrichment and LWR program far more robust than the US expected to exist, revealing the American belief that the status quo would stop North Korean nuclear advancement to be flawed \cite{snyder2}. Response to the Yeonpyeong and Cheonan incidents had further suggested that sanctions were not significantly more productive than disengagement \cite{delury}. Once again, having exhausted all other options and still uninterested in direct military conflict, the United States turned to diplomacy.

\section{Obama and Kim Jong-un, 2012-2013}

Kim Jong-il's death briefly threw negotiations into turmoil \cite{crs13}---perhaps because the US was waiting to see if the transition to Kim Jong-un's leadership would cause the collapse that many experts had been forecasting for years \cite{delury}. This did not occur, however, and the DPRK suggested resumption of talks, after which an agreement was announced on February 29, 2012 \cite{delury}. The United States agreed to provide significant quantities of food aid in exchange for a moratorium on nuclear activities and the return of IAEA inspectors \cite{crs13}.

Though the United States hoped that the Leap Day Agreement would forestall further nuclear provocations by the DPRK, announcement of a satellite launch a mere three weeks later (the beginning of cycle 10) suggested that these hopes were in vain \cite{snyder2}. The DPRK insisted that these actions were an exercise of their right to peaceful space exploration \cite{crs13}, and claimed to still be interested in implementing the Agreement. However, the US (and the international community \cite{unsc13}) saw this action as a provocative missile test \cite{davenport}. This meant that, to American eyes, the Leap Day Agreement had collapsed and the diplomatic approach was judged a wholesale failure \cite{delury}.

Following this ``failure'', the US began to push for more aggressive UNSC sanctions than had been seen previously during the Obama administration, and continued this when another ``satellite launch'' occurred at the end of the year \cite{crs13}. The DPRK proceeded to conduct a nuclear test in early 2013, once again facing severe sanctions from the UNSC \cite{davenport}. Whether out of frustration with the Agreement's inability to change North Korean behavior or due to internal pressure from conservatives in Congress, the Obama administration once again swung from a more pro-dialogue stance to one in favor of punishment and sanctions \cite{snyder2}.