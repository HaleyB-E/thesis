\chapter{Conclusions}

\section{The Agreed Framework and the Leap Day Agreement}

In tracing the history of American-North Korean nuclear diplomacy, it is worthwhile to compare the two major bilateral agreements between the nations - the Leap Day Agreement and the Agreed Framework. Though both ultimately collapsed, the differences in modes of failure are illustrative for the purposes of identifying successes and flaws in the American diplomatic model.

The Leap Day Agreement, like the Agreed Framework, promised aid from the United States in exchange for a freeze on the North Korean nuclear program. Unlike the Agreed Framework, which lasted for several years of spotty implementation only to peter out in the midst of a crisis, the Leap Day Agreement lasted less than a month. The difference was one of both circumstance and attitude.

The Agreed Framework took shape when both parties were actively seeking an escape from the escalatory cycle they were embroiled in. The DPRK was facing a flagging economy, strict international sanctions, and serious military threats from the United States. Meanwhile, the United States was unwilling to stomach the consequences of following through on the military threat, and was running out of options for producing economic pressure. Thus, both sides had strong incentives to cooperate with one another at least enough to weather the crisis.

During the negotiation of the Leap Day Agreement, however, the pressures felt by each party were far less symmetric. The DPRK was in a much stronger position than it had been in the 1990s. As a result of a decade mostly unfettered by the direct oversight of the international community (first due to Bush's aggressive disengagement from nonproliferation-focused agreements, then due to Obama's strategic patience), their nuclear program was robust enough that they no longer considered the American military an existential threat \cite{harnisch,hecker2}. Sanctions no longer seemed as effective a threat---perhaps because they were ``smart sanctions'' focused on weapons technology \cite{delury}, perhaps because the North Korean economy was in less dire straits, or perhaps because the international community was not following through on enforcing sanctions \cite{perlez}.

Meanwhile, the United States had come to the alarming conclusion that nothing else it tried had worked. ``Strategic patience'' relying on sanctions and other measures already in place seemed to have been based on a fundamental misestimation of the technical capacity of the North Korean nuclear program \cite{snyder2}. The sanctions didn't seem to be working, and the US was out of options---just like in 1994. This time, however, it was only American negotiators that were desperate.

The United States, then, placed a great deal of significance on the terms of the Leap Day Agreement, hoping strongly that it would succeed. The DPRK, on the other hand, was less concerned, which made them comfortable with pushing the boundaries of the Agreement. They took a sticking point from the Agreement's negotiations---whether satellite launches were legitimate enterprises or unacceptable tests of missile technology \cite{delury}---and demanded that their interpretation of the Agreement be accepted. They simultaneously affirmed willingness to hold up what they saw as their end of the bargain---the fact that they have previously shown no compunctions about declaring their departure from agreements suggests that this may even have been a sincere declaration.

Meanwhile, although the urgency of restraining the North Korean nuclear program would seem to have been a strong motivator in forcing the US to go to the bargaining table initially, it didn't seem to motivate them to attempt to salvage the Leap Day Agreement. Immediately upon the start of cycle 10, the US seemed to give up on the Agreement as a failure and indeed repudiated bilateral negotiations with the DPRK as a strategy wholesale \cite{delury}.

This may have been motivated by the belief that capitulating to North Korean interpretations of agreements would damage American credibility beyond repair---a view that seems to hold sway among conservative American politicians (who likely would have loudly criticized the Obama administration had any apparent capitulation occurred) and many of those involved in the UNSC sanctioning efforts. It seems bizarre to abandon the greatest opportunity in recent years to engage in actual negotiations over fear of diminished ability to negotiate. In general, however, this is consistent with much of the Obama administration's tactics---in the interest of avoiding taking any diplomatic tools out of the toolbox permanently, many opportunities to actually use the tools as intended have been lost.

In sum, the Agreed Framework lasted longer than the Leap Day Agreement because there was more buy-in from both sides at the start. Likewise, the Agreed Framework collapsed as this buy-in eroded on both sides. This exacerbated the North Korean concerns about American intentions that informed their unwillingness to make the first move. North Korean reticence, in turn, reinforced the American view that North Korea wasn't to be trusted. The Leap Day Agreement seemed more necessary to the US than it did to the DPRK, so when the US claimed that North Korean action had invalidated the Agreement, the DPRK did not have especially strong incentives to fight the collapse.

\section{Lessons Learned and Future Prospects}

There have been circumstances during which it has been possible to get the DPRK to agree to dial back its nuclear program. There have even been circumstances in which it has actually done so. The difficulty lies in identifying common threads that tie these circumstances together, and in attempting to recreate them.

American strategy in the wake of the collapse of the Leap Day Agreement has two primary components. The first eschews direct contact with the DPRK---a vestige of the ``strategic patience'' of the early Obama days, or perhaps a reaction to the ``failure of diplomacy'' evidenced by the Agreement's collapse. The second emphasizes construction of international pressure---through both global nonproliferation norms and direct sanctions from the UNSC and other bodies.

This bifurcated strategy seems at odds with the lessons of the historic record. Negative international pressure spearheaded by the United States has a notably poor track record, and the crises of the first four years of Obama's presidency highlighted the dangers of disengaging and leaving North Korea to its own devices.

\subsection{The Problem With International Pressure}

For North Korea to come to the table with denuclearizing intentions, it must feel vulnerable---pressured by both external and internal forces. The international community is sometimes a useful tool in creating this pressure, as demonstrated by the sanctions before the Agreed Framework and the brief success of the six-party talks' statements of principles. 

However, too much apparent curation of international pressure by the United States risks inciting anger from the DPRK, which is always vigilant for American conspiracies forcing it to act in certain ways. If the DPRK suspects collusion between the US and NGOs or countries involved in nuclear talks or implementation of agreements, it tends to balk, doubling down on escalatory behavior and inflammatory rhetoric to avoid what it sees as American imperialism. Thus, even though the existence of external pressure may encourage the DPRK to negotiate, cultivating this pressure in a productive manner is extremely difficult.

\subsection{The Problem With Strategic Patience}

If the DPRK was sufficiently hindered in development of nuclear and missile technology by the existence of safeguards, oversight, and existing sanctions, a policy putting the onus on North Korea to act first might work. However, that was not the case in 2009 and it has not been the case since then. North Korea has become more motivated and confident in its nuclear program with every passing year---and, with that, the chance of total denuclearization becomes more and more remote.

First, from a practical and ideological standpoint, the further North Korea progresses on its nuclear program, the less likely it is to want to denuclearize. A program in relative infancy would be relatively easy to shut down. A program with facilities capturing much of the weapons supply chain and with a number of successful tests under its belt, however, involves far more investment and as such would be commensurately harder to shutter. Greater material incentives would be necessary to offset the increased financial loss represented by closure. Also, if the DPRK's national image has been tied to the success of its program, the difficulty of convincing it to shut down the program increases substantially. This latter factor in particular has been noticeable in the DPRK's posture over time \cite{rich14,hecker2}.

Second, the gap between how much trust exists within the US-DPRK relationship and how much is necessary for denuclearization to work increases as the North Korean program advances. A more robust nuclear program means more facilities, more materials, and more missiles---and with this increase in numbers comes increased complexity for monitoring and drawdown efforts. Verification that complete denuclearization has occurred is a difficult problem even when all parties have a reasonable working relationship. As years pass and US-DPRK interactions continue to show suspicion or bad faith, the likelihood of a relationship in which denuclearization can be believed decreases as well.

Waiting for the DPRK to make the first move, then, is both theoretically unsound and contraindicated by historical facts. The DPRK is not under sufficient existential pressure to foster willingness to talk on American terms, and as long as the status quo persists, this means they will continue to advance their nuclear program---which will itself make future negotiations on denuclearizing terms less feasible.

\subsection{Is American Strategy Reasonable?}

Current American strategy is bifurcated between indirect engagement (through cultivating pressure via the international community) and direct disengagement (through unwillingness to make the first diplomatic overture). This is informed to some extent by historical facts---the DPRK has come to the bargaining table when it has felt sufficient pressure, and does seem more willing to stick to agreements for longer if it is the initiating party in negotiations. However, the conditions in the status quo are sufficiently far from the conditions under which these strategies have succeeded that they no longer seem likely to work.

The particular brand of pressure engendered by American advocacy of tough measures like increased UNSC sanctions is highly likely to make the DPRK double down on hard-line stances in order to prove it will not be bullied. The North Korean nuclear program is both robust and significantly linked with the nation's self-perception of international legitimacy. The only hope, then, would be to create sanctions that pose a sufficient existential threat to the DPRK that it has no choice but to negotiate---a level which, through accident or concern for civilian life, the UNSC has not reached\footnote{Perhaps some internal catastrophe like a series of natural disasters or the sudden death of Kim Jong-un could also produce such an existential threat, but aside from assassinations or the deliberate triggering of a volcanic eruption, this is beyond the ability of the United States to affect.}. 

This indicates that current American strategy - and, in particular, focus on extracting commitments to complete denuclearization before further talks can occur---is fundamentally flawed. Indeed, even if the precondition were dispensed with, it is hard to imagine a scenario in which the DPRK would seriously commit to abandoning its nuclear ambitions without dramatic shifts to the status quo. America must either radically reconsider what it is willing to do in order to remove North Korea as a nuclear threat, or face the reality that the DPRK is a nuclear power and shift its focus to damage control.

\todos{}