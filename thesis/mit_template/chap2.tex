\chapter{Background}

\section{Brief history of North Korean nuclear programs and negotiations}

Throughout the 1950s, the DPRK and the USSR worked together to initiate the North Korean nuclear power program, beginning construction on the Yongbyon nuclear complex in the early 1960s \cite{nti15}. The country became party to limited IAEA regulations in the following decade for the 5 MWe reactor constructed at Yongbyon, but did not develop a complete safeguards agreement until some time after its entrance into the NPT in 1985 \cite{iaea09}. During the 80s, North Korea continued to expand the Yongbyon complex and look into LWR technology \cite{ntiYongbyon}.

Over the next several decades, the US and the DPRK signed a number of agreements under which assurances were made about the provision of LWRs to North Korea in exchange for regulations intended to limit risk of proliferation \cite{agreed, davenport}. However, KEDO - the organization created to implement these agreements by constructing the LWRs - faced significant challenges in terms of funding and political support, eventually functionally collapsing under the strain of disagreements between the DPRK and member states in 2006 \cite{kedohist}.

Simultaneously, the Yongbyon reactors that the KEDO project was intended to replace existed in a constant state of flux. The complex would be shut down pursuant to some new agreement, or the IAEA would be allowed to inspect portions of the site\ldots but then, a few years later, the inspectors would be expelled and the facilities restarted again \cite{davenport,nti15,iaea09}.

As the reprocessing facility in Yongbyn operated in fits and starts, the DPRK gradually accumulated enough plutonium to produce nuclear weapons. An underground nuclear test in a DPRK facility in 2006 shocked and horrified the international community, which immediately responded with political admonishments and economic sanctions \cite{davenport}. To date, the DPRK has tested nuclear weapons three more times, along with a number of missile tests going back to 1998 \cite{orfall} and including launches in 2013 and 2016 that culminated in placement of a satellite into orbit \cite{davenport}. In addition to the nuclear tests, missile tests and rocket launches by the DPRK have been roundly criticized by the UN and many individual nations as attempts to develop effective nuclear missiles \cite{unApr12}.


\section{Current Political Status}

The North Korean nuclear program has been shut down and restarted a number of times, pursuant to various agreements made and then broken between the DPRK and a number of negotiating bodies \cite{bajoria,davenport}. The DPRK insists that its peaceful program should be allowed to continue \cite{kcna2}, while the US and South Korea claim that even this is unacceptable \cite{lee} - and demand complete commitment to denuclearization before any further negotiations can even begin.

This hard-line position is informed by fact - the Yongbyon reactor, nominally a peaceful source of power, did produce the material that has been used in the DPRK's weapons program \cite{hecker}. But the stance has proven unhelpful in getting North Korea to the table for negotiations - they claim that it is their right as a sovereign nation to pursue both peaceful and military nuclear technology, and that maintaining this right is a necessary deterrent against foreign aggression \cite{kcna,kcna2}.

Each side, then, is demanding as a precondition for negotiation the very action that the other side refuses to even consider. It is perhaps unsurprising that this has not been effective. The result has been a cycle of behavior that has gone on for several decades: aggression and posturing on North Korea's part is followed by sanctions and demands by other nations. Tension escalates until one side or the other begins to call for a diplomatic solution, which either ends in stalemate or produces results that only last for a short time before the agreements are broken again \cite{bajoria, davenport}.

\section{Previous Analysis of the Problem}

Many analyses of the nature of the politics surrounding DPRK nuclear diplomacy exist, and several have concluded that the problem is cyclic in nature \cite{blair,cfr,fisher,gause,habib,jun}. In general, the issue tends to be modeled as a series of cycles comprised of initial posturing, followed by aggression and escalation, and concluding with reconciliation that, depending on the analyst, may or may not be identified as sincere.

When these analyses are politically-motivated \cite{blair, cfr}, their logic tends to begin with the initial proposition that demand for complete denuclearization can someday be fulfilled, if only the correct combination of diplomacy and coercion can be found that will force the DPRK to comply. More academic treatments of the problem \cite{habib,jun} vary widely in their approach. Jun \cite{jun} identifies problems with past approaches, but seems to conclude that increased coordination between negotiating bodies and oversight of agreement implementation may yet salvage the denuclearization agenda. Meanwhile, Habib \cite{habib} claims that North Korea's cyclic behavior is inevitable due to the nation's military-first ethos. He further postulates that the nuclear program is a key part of this ethos, and thus that the DPRK will never sincerely agree to give it up.

Though the concept of modeling US-DPRK relations as a series of cycles is well-supported, the form this model takes within the literature leaves something to be desired. Often, for the sake of simplicity, pithiness, or with the hope of revealing some greater structure within a complicated historical progression, the model will postulate that all important US-DPRK negotiations can be seen as a repetition of a single process, three to five steps long \cite{fisher,jun}. It is possible - and, indeed, likely - that this is an underfitting that loses more in its lack of subtlety than it gains in elegance.

There is a decently broad body of research focusing on rhetorical analysis of the North Korean nuclear issue. Several authors \cite{rich12, rich14, sin} have worked with automated content analysis to look at how references to countries and specific incidents correlate with references to nuclear programs in official North Korean news outlets.

Additionally, work on the shifts and nuances of American rhetoric in nuclear negotiations with the DPRK exists - albeit in somewhat less detail than the work done on KCNA documents \cite{bleiker,cumings,harnisch,huntley}. However, it does not appear that any kind of large-scale data-aggregating study has been done on American official statements in the manner of studies on North Korean documents. This makes sense given the relative volume of official material produced on the issue by each government, but complicates the process of comparing American and North Korean rhetorical strategies.