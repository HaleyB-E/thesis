\chapter{Historical Foundation}
\section{Methodology}

Identifying the timing and relationship between various events---like sanctions, treaty negotiations, and missile tests---may illuminate consistency that will allow conclusions to be drawn about shifts in North Korean diplomatic posture. These correlations may indicate consistently successful (or unsuccessful) strategies on the part of North Korea, and they may provide insight as to the efficacy of political tools of those who seek to change the DPRK's behavior.

In keeping with established literature, analysis is patterned on the notion that US-DPRK relations are cyclic in nature. However, unlike much previous work, no attempt is made to fit individual behavior cycles to specific models. Instead, a broad analysis of instances of escalatory behavior is undertaken. Since this does not rely upon fitting all categories of events into a single three-to-five-step process (as is common within historical analysis of this problem \cite{fisher, jun}), it may be possible to identify correlations that have previously been missed.

Thus, escalatory cycles are here defined as any instigating event followed by a chain of other events which are direct responses to some action earlier in the chain. This allows classification into individual cycles for ease of pattern analysis, while avoiding the preconceptions built into a model with defined steps.

Analysis will be limited to events occurring before 2014. This does mean that conclusions cannot take the most recent events into account - however, initial examination suggests that the events of 2014 and later are part of an escalatory cycle that is still ongoing. It seems unproductive to attempt to draw conclusions about the results of escalatory cycles using data from a cycle that is still unfolding.

Even a model of this generality will exclude some details and significant events in US-DPRK diplomatic history. However, the purpose of this analysis is not a complete encapsulation of information about negotiations. Rather, the intention is to narrow a complicated multilateral diplomacy problem to a more tractable model, by forming a timeline of significant events that will serve as the backbone for subsequent analysis of technical and rhetorical changes.

\section{Timeline}
\subsection{Early development of the North Korean nuclear program}
In 1952, North Korea established an organization to begin research into nuclear technologies, the Atomic Energy Research Institute \cite{ntiAERI}. Four years later, with little progress made, they signed an agreement with the USSR in order to train Korean scientists \cite{nti15}. Then, in 1959, additional agreements gave birth to the Yongbyon research complex, as the USSR began to assist the DPRK with construction and materials for a research reactor in addition to technical training \cite{nti15}.

By the early 1970s, nuclear expertise in the DPRK had advanced to the point that they were no longer reliant upon the Soviet Union for reactor technology (and had begun to make their own improvements on the IRT-2000 research reactor design) \cite{nti15}. However, the partnership was still strong, and the USSR began to provide the DPRK with assistance in developing plutonium reprocessing capacity \cite{nti15}.

A decade later, North Korea's technical development continued to improve rapidly as they pursued technologies across the nuclear fuel cycle, as well as several significantly larger reactors (5 MWe in 1979 \cite{ntiYongbyon}, 50 MWe in 1986 \cite{ntiYongbyon2}). Simultaneously, they began to pursue light water reactors (LWRs), and signed the Treaty on the Non-Proliferation of Nuclear Weapons on the condition that they be provided additional construction assistance \cite{nti15}.   

\subsection{Cycle 1}

In 1992, provisions of the IAEA Safeguards Agreement associated with membership in the NPT went into effect, and the DPRK filed an initial report declaring the contents of their nuclear inventory \cite{iaea92}. However, IAEA analysis suggested a notably higher level of plutonium should exist in DPRK waste streams and stockpiles than was declared, and requested access to waste sites in order to verify or disprove the existence of undeclared plutonium \cite{iaea09}. 

Rather than accede to this request, the DPRK refused access, claiming that the sites were military in nature and thus exempt from the Safeguards Agreement \cite{nti15,iaea09}. In response, the IAEA requested authorization from the UNSC to perform special inspections on those sites \cite{nti15}.
 
In response to the IAEA's request, North Korea declared it was withdrawing from the NPT in order to protect ``the supreme interests of its country'', effective 90 days after the declaration as per Article X of the treaty \cite{npt}. The United States hastily began bilateral talks, and the day before the withdrawal was to come into effect, the DPRK announced that it was suspending this action for at least until negotiations were completed \cite{nti15}. The result of these negotiations, announced in July 1993, was a joint statement between the United States and North Korea. In this statement, the United States agreed to ``support the introduction of LWRs'' into North Korea to replace their existing graphite-moderated reactors, while the DPRK agreed to ``begin consultations with the IAEA on outstanding safeguards and other issues as soon as possible'' \cite{hayes}. 

\subsection{Cycle 2}

In February of 1994, the agreement with the IAEA prescribed by earlier negotiations was finalized, and inspectors were allowed back into several of the DPRK's nuclear facilities \cite{davenport}. However, the inspections that were permitted to occur were incomplete, as the DPRK insisted that only ``continuity of safeguards'' was required \cite{iaea09}. Under this paradigm, IAEA inspectors were only allowed into areas they had already been permitted to access---further, they were not allowed to take additional actions that might verify or disprove the existence of undeclared plutonium stores. This angered the IAEA Board of Governors, who believed that they could not accurately determine whether proliferation-related programs were occurring under such conditions \cite{davenport}.

In May, the IAEA's concerns were realized, as the 5 MWe Yongbyon reactor's fuel rods were removed without supervision and stored without preserving details of their previous locations within the core. In doing this, the DPRK made it impossible to usefully examine the fuel rods to look for evidence indicating that some had been removed for plutonium production when inspectors had not been present \cite{nobacksies}. This caused the IAEA Board of Governors to release a statement condemning the DPRK's actions and suspending all non-medical assistance to the country by the IAEA \cite{iaea94}.

In response, the DPRK withdrew its membership from the IAEA. As a party to the NPT, the IAEA claimed it was still subject to the existing safeguards agreements - but North Korea disagreed, and refused to allow inspectors into its nuclear facilities \cite{iaea09}. Tensions continued to rise, to the point that the United States began seriously considering air strikes on Yongbyon \cite{jun}. Eventually, however, the crisis was defused as Jimmy Carter traveled to the DPRK and began negotiations that eventually culminated in the Agreed Framework \cite{nti15}. The DPRK agreed to freeze the Yongbyon graphite reactors and its reprocessing program, while the US made more concrete guarantees regarding provision of LWR technology and energy aid to offset the impact of the reactor freeze \cite{agreed}. By November, the IAEA was able to confirm that operations on Yongbyon had ceased \cite{davenport}.

\subsection{Cycle 3}

In August of 1998, the DPRK launched a Taepodong rocket in a (likely unsuccessful) attempt to carry a satellite into orbit\cite{orfall}. Many nations denounced this as an unacceptable missile test, and were concerned by the advances in range and complexity of North Korean missile technology that it displayed\cite{orfall}. As a result, Japan suspended diplomatic talks and considered halting its funding for the Agreed Framework's LWR program\cite{orfall}.

Negotiations between the United States and North Korea began again, but were largely unsuccessful---the US offered sanction relief in exchange for termination of the DPRK missile program, but the latter claimed that sanction relief was already part of the Agreed Framework and thus not a valid incentive for negotiation\cite{davenport}. In various iterations, negotiations continue until September 1999, when the DPRK agreed to temporarily refrain from conducting long-range missile tests in exchange for limited sanction relief\cite{davenport}.

\subsection{Cycle 4}

In late 2002, an American official in North Korea for talks mentioned a variety of concerns held by the US regarding the DPRK's record on nuclear proliferation and human rights \cite{davenport}. He suggested that the DPRK should work on these issues in order to improve relations with the United States, which was taken by DPRK officials as “high handed and arrogant” policy which placed unreasonable unilateral demands on them \cite{kcna3}. In this already-tense environment, conditions worsened when the US official told North Korean representatives that the United States knew about a secret enrichment facility, then publicly claimed that the North Korean representatives confirmed its existence \cite{davenport}. 

The DPRK denied any such admission, but the United States (along with other countries involved in the LWR construction program) declared that sufficient evidence of violation of several treaties existed that they were suspending the energy aid negotiated under the Agreed Framework \cite{iaea09}. The IAEA attempted to gather more facts on the situation, in a manner regarded by the DPRK as ``acting under the manipulation of the United States'' \cite{hurriyet}---so, in response, the DPRK began removing IAEA seals, expelling inspectors, and generally restarting its graphite reactor program \cite{iaea09}.

The IAEA Board of Governors was extremely displeased with this turn of events, and released a statement condemning the DPRK's behavior \cite{iaea03}. In response, the DPRK withdrew from the NPT, claiming that they could bypass the requisite 3-month waiting period because their withdrawal in 1993 had simply been temporarily suspended \cite{kcna4}. The IAEA expressed ``deep concern'' and referred the issue to the UNSC, which also expressed ``concern'' \cite{iaea09}. However, such concern did not stop the Yongbyon 5 MWe reactor from being restarted, which occurred in February 2003 \cite{davenport}. Over the next several months, talks between the US, the DPRK, and China occurred to little effect as reactor operation and spent fuel reprocessing continued apace \cite{davenport}.

\subsection{Cycle 5}
In September of 2005, the United States froze North Korean funds in Banco Delta Asia, citing money-laundering concerns, association with drug trafficking, and suspected US currency counterfeiting \cite{davenport}. Banco Delta Asia was designated a ``primary money-laundering concern'' and the bank was prohibited from doing business in US dollars. Immediately, other banks around the globe began to refuse to do business with the DPRK, fearing similar reprisals \cite{greenlees}.

The next major round of six-party talks began shortly thereafter, and the DPRK delegation focused on the issue of the bank freeze to the exclusion of other issues \cite{greenlees}. As a result, little progress was made and the talks stalled. In early 2006, the US Treasury Department and DPRK officials discussed ways to resolve the Banco Delta Asia conflict, but remained at a stalemate---the DPRK would return to talks if the funds were unfrozen, but the US wanted to discuss issues related to the funds in multilateral negotiations \cite{greenlees}.

\subsection{Cycle 6}

During the summer of 2006, North Korea fired a number of missiles, including a long-range Taepodong-2. In response, South Korea halted aid programs, Japan imposed sanctions, and the UNSC sanctioned missile-related technology \cite{greenlees}. Additionally, the UNSC resolution urged a return to the six-party talks and North Korea's previous (voluntary) missile test moratorium \cite{unsc06}. The DPRK ``vehemently denounce[d] and roundly refute[d]'' the resolution, vowing to ``bolster its war deterrent for self-defense'' in whatever ways it saw fit \cite{kcna5}.

This threat was made good in October, when the DPRK conducted its first nuclear test \cite{nti15}. Though the test was likely not particularly successful, it still shocked the international community---the UNSC responded with additional sanctions and demands that North Korea roll back its nuclear program, avoid any further testing, and return to IAEA oversight \cite{unsc1718}. The six-party talks resumed in November in an apparent victory for diplomatic efforts, but due to lingering disagreements over Banco Delta Asia and North Korea's unwillingness to work ``unilaterally'', the talks concluded without result at the end of the year \cite{davenport}.

The talks resumed in February, 2007---this time, resulting in substantive agreements. The DPRK promised to return to the NPT and IAEA surveillance, and to shut down and seal the Yongbyon reactors and other nuclear facilities. In exchange, they would receive a sizable food and energy aid package \cite{js5}. Enactment of this agreement was briefly stalled over continuing concerns related to the Banco Delta Asia funds, but the United States eventually agreed to unfreeze them and in return the DPRK began to shut down the Yongbyon facility under IAEA supervision \cite{davenport}.

Under another agreement from the most recent round of six-party talks, the DPRK was supposed to submit a declaration of its nuclear programs by the end of 2007 \cite{js6}. It did not do so, due (according to the US State Department) to ``some technical questions'' \cite{seanm}. The statement was finally released in June, and despite concerns about the completeness of the document, the United States announced its intention to remove the DPRK from the State Sponsors of Terror list, as well as removing some sanctions and other trade barriers \cite{nti15}.

When the United States failed to remove the DPRK from the state Sponsors of Terror list after the 45-day waiting period had expired, the DPRK announced that it would halt demolishment of its graphite reactors and was willing to begin construction again \cite{davenport}. By September 2008, the DPRK had asked the IAEA to remove its seals on the North Korean reprocessing plant, and IAEA officials warned that the DPRK intended to begin reprocessing material shortly \cite{iaea09}. The US hastily reopened negotiations, and reached an agreement whereby the State Sponsors of Terror delisting would occur in exchange for a return to disablement \cite{nti15}.

\subsection{Cycle 7}

Speculations about a North Korean missile launch began in February 2009. Several nations released statements to the effect that such a launch would violate a UNSC resolution and therefore the DPRK should not expect to be able to go forward without serious consequences \cite{davenport}. When the DPRK warned international organizations of the time and likely location of rocket stage splashdowns, the ROK announced that it was considering joining the PSI, a US-started effort to limit transport of weapons of mass destruction \cite{davenport}.

North Korea did indeed launch a rocket---likely a modified Taepodong-2 long-range missile, and allegedly for the purposes of launching a satellite \cite{davenport}. The UNSC released a statement condemning the launch and calling for a revisiting and strengthening of sanctions, while urging the DPRK to return to six-party talks \cite{unsc09}. In response, the DPRK withdrew from the six-party talks and a number of its agreements with the United States \cite{niksch}, removed IAEA safeguards and ejected inspectors \cite{iaea09}, and resumed construction on the mothballed 5 MWe reactor at Yongbyon \cite{nti15}.

In May, the DPRK conducted another nuclear test, likely somewhat more successful than its first \cite{nti15}. The UNSC convened an emergency meeting and released a Presidential Statement condemning North Korean nuclear tests and recommending the strengthening of sanctions \cite{unsc09p}. The ROK made good on its threat to join the PSI - which North Korea declared an act of war, voiding the Korean War armistice \cite{glionna}. The UNSC passed another resolution once again condemning North Korean nuclear tests and strengthening arms embargoes \cite{unsc09j}. This was immediately followed by a statement by the DPRK in which they outlined responses to the resolution, including increased attempts at uranium enrichment \cite{nti15}.

\subsection{Cycle 8}

In March of 2010, an ROK patrol ship, the ROKS Cheonan, sank near the Korean maritime border \cite{branigan}. Though the ROK initially refused to speculate on whether the DPRK was involved with the sinking \cite{branigan}, the South's government refused to negotiate with the North until the incident could be investigated \cite{davenport}. Analysis pointing to a deliberate torpedoing by the DPRK quickly emerged \cite{reuters}, and the ROK formally announced that it would sever most economic ties with the North as a result. The next day, the DPRK announced that it would ``cut all links'' to the South in response to the accusations \cite{davenport}. The United States imposed additional sanctions on the DPRK, and held a joint military exercise with the ROK ``to demonstrate the alliance's resolve'' and ``send a strong message to Pyongyang'' \cite{starr}.

\subsection{Cycle 9}

In November, barely half a year after the Cheonan torpedo incident, the DPRK shelled Yeonpyeong Island and the ROK returned fire \cite{bbc}. China called for an immediate return to the six-party talks \cite{bbc}, but several other participant states rejected on the grounds that relations between the DPRK and the ROK were not good enough for that to be reasonable \cite{davenport}.

In early 2011, the DPRK informed a Russian official that it would consider resuming the six-party talks, but the ROK rejected this offer on the grounds that there was no reason to believe in the sincerity of the DPRK's negotiating efforts \cite{davenport}. In May, the ROK offered the DPRK a position at the Nuclear Security Summit the following year, if they would commit to denuclearization---however, the DPRK denounced this as a ruse attempting to soften the North up for invasion \cite{davenport}. Over the course of the summer, the tone of diplomacy became markedly more positive, and resumption of the six-party talks appeared more and more likely \cite{davenport}.

\subsection{Cycle 10}

In March of 2012, the DPRK announced that it would launch a satellite the following month to commemorate the centennial of Kim Il-sung---which, according to the United States, would violate the terms of the Leap-Day Agreement signed barely a month previously \cite{davenport}. Shortly thereafter, the US temporarily suspended its delivery of food aid, which was then halted completely after the satellite launch was (yet again, unsuccessfully) attempted in April \cite{davenport}.

Although the UNSC quickly condemned the launch as a violation of numerous previous resolutions regarding ballistic missile launches \cite{unsc12}, the DPRK announced its intention to try again with a similar configuration later that year. In mid-December, the launch was attempted and, for the first time, external sources confirmed that the satellite achieved orbit \cite{davenport}. In response, the UNSC passed another resolution reaffirming previous sanctions and condemnations and demanding that North Korea ``abandon all nuclear weapons and nuclear programmes completely, verifiably, and irreversibly.'' \cite{unsc13}

In response, the DPRK announced that it intended to continue with missile testing, and additionally that it would soon conduct another nuclear test \cite{davenport}. Seismic activity consistent with underground nuclear detonation was detected in North Korea in early February of 2013 \cite{davenport}. The UNSC responded with another resolution ``strengthening and expanding the scope of'' existing sanctions \cite{unsc13m}.

\section{Summary Tables}

\begin{table}[H]
	\caption{Historical Cycles: Beginnings and Endings}
	\small
	\begin{tabular}{|l|p{5.75cm}|p{6cm}|}
	\hline
	Cycle (year) & Initiating Incident & End result \\ 
	\hline
	1 (1992--3) & Diplomatic engagement (DPRK files IAEA report) & US-DPRK joint statement---DPRK gains LWR tech assurances, promises to engage with IAEA \\ 
	\hline
	2 (1994) & Diplomatic engagement (DPRK allows IAEA inspectors again) & US-DPRK Agreed Framework---DPRK gains LWR tech assurances, energy aid; freezes graphite reactor programs \\ 
	\hline
	3 (1998--9) & Missile launch (DPRK attempts to place satellite into orbit) & US-DPRK agreement---temporary ban on long-range missile tests in exchange for sanction relief \\ 
	\hline
	4 (2002--3) & US action (official claims DPRK has a secret enrichment program) & Inconclusive negotiations; DPRK admits it has nuclear weapons \\ 
	\hline
	5 (2005--6) & US action (freezing DPRK assets in a Macau bank) & Major impasse to negotiations created; DPRK loses access to \$25 million in funds \\ 
	\hline
	6 (2006--8) & Missile launch (DPRK tests several missiles of different types) & 6-party agreement---rollback of reactor program, removal of DPRK from State Sponsors of Terror list, unfreezing of Banco Delta Asia funds \\ 
	\hline
	7 (2009) & Missile launch (DPRK attempts to put satellite into orbit) & Increase in sanctions, inconclusive negotiations \\ 
	\hline
	8 (2010) & Torpedo attack by DPRK & US sanctions; US-ROK joint military exercises \\ 
	\hline
	9 (2010--11) & Artillery shelling by DPRK & Inconclusive but positive-leaning negotiations \\ 
	\hline
	10 (2012--13) & Missile launch (DPRK attempts to put satellite into orbit) & Increase in sanctions, DPRK nuclear testing \\
	\hline
	\end{tabular}
\end{table}

\begin{table}[H]
	\caption{Historical Cycles: Notable Events} 
	\small
	\begin{tabular}{|c|p{3cm}|p{2.25cm}|p{6.5cm}|}
	\hline
	Cycle (year) & Nuclear/Missile tests & Sanctions & Changes to status quo \\ 
	\hline
	1 (1992--3) & None & None & DPRK gains assurances of LWR assistance in exchange for returning to previous agreements \\ 
	\hline
	2 (1994) & None & IAEA Board of Governors & DPRK gains assurances of LWR assistance and energy aid in exchange for returning to previous agreements \\ 
	\hline
	3 (1998--9) & Long-range missile & None & DPRK gains reduction in sanctions in exchange for halting already-sporadic tests for a short but unspecified time \\ 
	\hline
	4 (2002--3) & None & None & DPRK loses plausible deniability and US energy aid \\ 
	\hline
	5 (2005--6) & None & United States & DPRK loses access to funds\\ 
	\hline
	6 (2006--8) & Both & UN, US, other nations & Many sanctions imposed, others removed; DPRK loses access to reactor program, gains some aid \\ 
	\hline
	7 (2009) & Both & UN & Sanctions/arms embargoes imposed; DPRK regains access to reactor; ROK joins PSI \\ 
	\hline
	8 (2010) & None & US, ROK & Severance of trade/relations between DPRK and ROK; increase in US sanctions \\ 
	\hline
	9 (2010--11) & None & None & No major change---some increase in discussion about resuming six-party talks \\ 
	\hline
	10 (2012--13) & Both & UN & Multiple rounds of UN sanctions imposed \\
	\hline
	\end{tabular} 
\end{table}

\section{Results and Trends}

In early cycles, the DPRK tended to come out of the cycle having gained either direct material benefit (food, oil) or assurances of future benefits from other countries. In exchange, they made agreements that, for the most part, would have them roll back their nuclear program to the state that it was in prior to the beginning of the escalatory cycle. This suggests that they at least initially viewed their nuclear program as a useful negotiating chit for soliciting aid from the international community.

However, as time passed, the international community became less willing to accommodate the DPRK's behavior---perhaps because they learned from previous experiences in which the DPRK did not follow through on its commitments, or perhaps because the DPRK's actual behavior exceeded some threshold of unacceptable provocation. Regardless of cause, negotiations with the DPRK in recent years have involved more stringent demands and strict preconditions created by the US and other nations seeking to avoid the cyclic behavior exhibited in the past.

The historic record seems to show that negotiating with North Korea is most successfully done when an escalatory cycle begins with DPRK overtures. This is unsurprising---the DPRK comes to the table intentionally, without apparent coercion to justify unproductive posturing. In general, cycles in which the US initiated engagement resulted in markedly worse outcomes than cycles with DPRK-initiated diplomacy and even some cycles the DPRK instigated with missile tests. Since both US-initiated cycles began with hostile overtures, this cannot actually be taken to suggest that the US can never expect useful outcomes from cycles it initiates. However, it is certainly a sobering indicator that if the US wishes to initiate dialogue, it must be careful about how it goes about doing so.
