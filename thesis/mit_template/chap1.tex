\chapter*{Introduction}
\addcontentsline{toc}{chapter}{Introduction}

\section*{Motivation}
\addcontentsline{toc}{section}{Motivation}

The specter of North Korean nuclear weapons has haunted the world since the early days of their nuclear reactor program in the 1960s \cite{pincus}. The international community, in particular the United States and South Korea, has always been adamant that a nuclear-armed North Korea is intolerable \cite{kerry,parksk}.

In the past, many believed that it was possible to achieve peaceful nuclear power for the DPRK without undue risk of weapon production. Indeed, the 1994 Agreed Framework outlined a plan under which the United States would provide North Korea with light-water reactors \cite{agreed}. However, in recent years, even ``proliferation-resistant'' reactors like the LWRs have been deemed too great of a risk: In 2008, Lee Myung-bak ran his presidential campaign in South Korea on the founding principle that aid to the North should be dependent upon complete denuclearization \cite{snyder}. As of late 2015, the United States held the official position that resuming multilateral negotiations with the DPRK could not occur until denuclearization was guaranteed \cite{pennington}. North Korea, meanwhile, is insistent that it has the right to nuclear materials for both power plants and weaponry \cite{kcna}. 

A variety of methods have been tried to resolve this disagreement, from sanctions to aid packages to mandated denuclearization as a condition on resumption of treaty talks \cite{bajoria,davenport}. These methods have, at best, resulted in temporary disarmament or a partial rollback of the DPRK's nuclear programs - but, in every case, the changes were only temporary and North Korea eventually returned to developing its nuclear weapons technology \cite{davenport,nti15,iaea09}.

When these methods have not succeeded, the UN and individual countries attempting to change the DPRK's behavior have not fundamentally altered their approach \cite{davenport,nti15}. If negotiators believed that the strategies were flawed or based in a misunderstanding of the DPRK's nuclear situation, they would likely have sought different methods. Since they have instead repeatedly applied similar strategies over many years, it is reasonable to claim that they believe the assumptions underpinning negotiation strategies to be correct. It is worthwhile to investigate whether this belief is warranted.

\section*{Objectives}
\addcontentsline{toc}{section}{Objectives}

This thesis hypothesizes that United States strategy in negotiating with North Korea is inconsistent with the technical facts of the North Korean nuclear program and past negotiating behavior, and further that the cyclical nature of negotiations persists because negotiators base their positions on what they want to be true rather than the facts of the situation. 

In order to test this hypothesis, the history of negotiations from 1992--2013\footnote{Limiting analysis to an endpoint several years in the past allows this thesis to draw upon an established body of research - few papers have been written on the US-DPRK relationship during the second term of Obama's presidency. Additionally, as will be addressed in the historical section, events in the most recent 3 years appear to belong to a cycle that has not yet concluded, making analysis of these events less productive.} will be examined in conjunction with analysis of the rhetorical trends of both governments. Comparing this analysis with the technical advancements of North Korea's nuclear program helps form a more complete picture of North Korean objectives and abilities. A timeline of events and changes in rhetorical posture gives a picture of the evolution of the diplomatic engagement over time. Particular attention will be paid to the events surrounding the creation and collapse of the Agreed Framework and the Leap Day Agreement, two major bilateral agreements signed between the US and the DPRK. This will illuminate the current political trajectory and allow conclusions to be drawn about its viability.

A number of likely indicators of problems in American negotiation strategies can be predicted even before significant analysis has taken place. North Korea is unlikely to acquiesce to external pressure to reverse course on development of a nuclear arsenal and infrastructure if it is not provided with sufficiently valuable incentives---rolling back its nuclear trajectory would be quite costly, both financially and politically. Also, if its rhetoric regarding denuclearization has been consistent across a wide variety of incentives and penalties, it is unlikely that variation in the terms of agreements along existing strategies will have a noticeable impact.

Meanwhile, there are also some indicators that American negotiation strategies hold the potential for success: if, for example, there have been previous agreements which, even if ultimately unsuccessful (as most, if not all, have been), provided some sort of incremental improvement to the situation that lasted beyond the agreement's end, then that agreement had some utility. Also, if North Korea's technical capacity is still far from its desired endpoint and seems unlikely to reach that endpoint for many years, there is likely to be more room for productive negotiations.


