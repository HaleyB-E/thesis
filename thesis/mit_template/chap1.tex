\chapter{Introduction}

\section{Motivation}

The specter of North Korean nuclear weapons has haunted the world since the early days of their nuclear reactor program in the 1960s \cite{pincus}. The international community, in particular the United States and South Korea, has always been adamant that a nuclear-armed North Korea is intolerable \cite{kerry,parksk}.

In the past, many believed that it was possible to achieve peaceful nuclear power for the DPRK without undue risk of weapon production – indeed, the 1994 Agreed Framework outlined a plan in which the United States would help North Korea develop a light-water reactor program \cite{agreed}. However, in recent years, even ``proliferation-resistant'' reactors like the LWRs have been deemed too great of a risk. In 2008, Lee Myung-bak ran his presidential campaign in South Korea on the founding principle that aid to the North should be dependent upon complete denuclearization \cite{snyder}. As of late 2015, the United States held the official position that resuming multilateral negotiations with the DPRK could not occur until denuclearization was guaranteed \cite{pennington}. North Korea, meanwhile, is insistent that it has the right to nuclear materials for both power plants and weaponry \cite{kcna}. 

A variety of methods have been tried to resolve this disagreement, from sanctions to aid packages to mandated denuclearization as a condition on resumption of treaty talks \cite{bajoria,davenport}. These methods have, at best, resulted in temporary disarmament or a partial rollback of the DPRK's nuclear programs - but, in every case, the changes were only temporary and North Korea eventually returned to developing its nuclear weapons technology \cite{davenport,nti15,iaea09}.

When these methods have not succeeded, the UN and individual countries attempting to change the DPRK's behavior have not systematically altered their approach \cite{davenport,nti15}. This indicates that negotiators believe the strategies themselves are valid, and grounded in a correct understanding of the DPRK's nuclear situation. It is worthwhile to investigate whether this assumption is warranted.

\section{Objectives}

It is hypothesized that United States strategy in negotiating with North Korea is inconsistent with the technical facts of the North Korean nuclear program and past negotiating behavior. In other words, the cyclical nature of negotiations persists because negotiators base their positions on what they want to be true about the DPRK, rather than the facts of the situation. 

In order to test this hypothesis, the history of negotiations from 1992-2013\footnote{Limiting analysis to an endpoint several years in the past allows this thesis to draw upon an established body of research - few papers have been written on the US-DPRK relationship during the second term of Obama's presidency. Additionally, as will be addressed in the historical section, events in the most recent 3 years appear to belong to a cycle that has not yet concluded, making analysis of these events less productive.} will be examined in conjunction with analysis of the rhetorical trends evident within the administrations of both governments. Underpinning this analysis with a firm understanding of the technical advancements in North Korea's nuclear program will form the most complete picture of their objectives and abilities. The results of examining the timeline of events and the changes in rhetorical posture will allow development of a picture of the evolution of diplomatic engagements over time. Particular attention will be paid to the events surrounding the creation and collapse of the Agreed Framework and the Leap Day Agreement, two major bilateral agreements signed between the US and the DPRK. This will illuminate the current political trajectory and allow conclusions to be drawn about its viability.

A number of likely indicators of problems in American negotiation strategies can be predicted even before significant analysis has taken place. If North Korea's technical achievements are on trajectory to match its stated end goals for a nuclear arsenal and infrastructure, it seems unlikely to acquiesce to external pressures to reverse course. Also, if its rhetoric regarding denuclearization has been consistent across a wide variety of incentives and penalties, it is unlikely that variation in the terms of agreements along existing strategies will have a noticeable impact.

Meanwhile, there are also some indicators that American negotiation strategies hold the potential for success: if, for example, there have been previous agreements which, even if ultimately unsuccessful (as most if not all have eventually been), provided some sort of incremental improvement to the situation that lasted beyond the agreement's end. Also, if North Korea's technical capacity is still far from their desired endpoint and seems unlikely to reach it for many years, there is likely to be more room for productive negotiations.


