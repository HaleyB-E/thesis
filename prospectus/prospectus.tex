\documentclass[titlepage]{article}
\usepackage[utf8]{inputenc}
\usepackage{tabu}


\title{Prospectus: Rethinking American Negotiation Strategies in North Korean Nuclear Talks}
\author{Haley Brandt-Erichsen}
\date{December 2015}

\begin{document}
\begin{titlepage}
    {\centering
    {\bf \Large Massachusetts Institute of Technology Department of Nuclear Science and Engineering \par}
    \vspace{1.5cm}
    {\large Thesis Prospectus \par for the \par Bachelor of Science Degree \par in \par Nuclear Science and Engineering \par December 2015 \par}
    \vspace{2cm}
    {\bf \LARGE Rethinking American Negotiation Strategies in North Korean Nuclear Talks \par}}
    
    \vfill
    
    {Author: \_\_\_\_\_\_\_\_\_\_\_\_\_\_\_\_\_\_\_\_\_\_\_\_\_\_\_\_\_\_\_\_\_\_\_\_\_\_\_\_\_\_\_\_\_\_\_\_\_\_\_\_\_\_\_\_\_\_\_\_\_\_\_\_\_\_\_\_\_\_\_\_\_\_\_\_\_\_\_\_\_}
    \begin{flushright}
        Haley Brandt-Erichsen \par
        Department of Nuclear Science and Engineering
    \end{flushright}
    \vspace{0.5 cm}
    {Certified by: \_\_\_\_\_\_\_\_\_\_\_\_\_\_\_\_\_\_\_\_\_\_\_\_\_\_\_\_\_\_\_\_\_\_\_\_\_\_\_\_\_\_\_\_\_\_\_\_\_\_\_\_\_\_\_\_\_\_\_\_\_\_\_\_\_\_\_\_\_\_\_\_\_\_\_\_\_\_\_}
    \begin{flushright}
        R. Scott Kemp \par
        Assistant Professor of Nuclear Science and Engineering \par
        Thesis Supervisor
    \end{flushright}
    \vspace{0.5 cm}
    {Accepted by: \_\_\_\_\_\_\_\_\_\_\_\_\_\_\_\_\_\_\_\_\_\_\_\_\_\_\_\_\_\_\_\_\_\_\_\_\_\_\_\_\_\_\_\_\_\_\_\_\_\_\_\_\_\_\_\_\_\_\_\_\_\_\_\_\_\_\_\_\_\_\_\_\_\_\_\_\_\_\_}
    \begin{flushright}
        Michael P. Short \par
        Assistant Professor of Nuclear Science and Engineering \par
        Chairman, NSE Committee for Undergraduate Students
    \end{flushright}
    
    
\end{titlepage}

%\maketitle
\newpage
\thispagestyle{empty}
       \addtocounter{page}{-1}
       \null
\ \newpage


\section{Introduction}
The specter of North Korean nuclear weapons has haunted the world since the early days of their nuclear reactor program in the 1960s \cite{pincus}. The international community, in particular the United States and South Korea, has always been adamant that a nuclear-armed North Korea is intolerable. In recent years, even nuclear power plants have been deemed an unacceptable risk \cite{kerry,lee}. North Korea, meanwhile, is very insistent that it has the right to nuclear materials for both power plants and weaponry \cite{kcna}. A variety of methods have been tried to resolve this disagreement, from sanctions and aid restrictions to mandated denuclearization as a condition on treaty talks \cite{bajoria,davenport}.

These methods have, at best, resulted in temporary disarmament or a partial rollback of the North Korean nuclear programs – but, in every case, the changes were only temporary and North Korea eventually returned to developing its nuclear weapons technology \cite{davenport,nti15,iaea09}. However, the UN and individual countries attempting to change North Korean behavior have responded to these failures by continuing to use the same kinds of tactics \cite{davenport,nti15}. This indicates that negotiators believe that the strategies themselves are valid, and grounded in a correct understanding of the North Korean nuclear situation. It is worthwhile to investigate whether this assumption is warranted.

\section{Background}
The prospect of North Korea with nuclear weapons alarms people. While negotiations in the past promoted the possibility of peaceful nuclear power for the country, in recent years it has been the predominant negotiating position that even this is too risky. This is a position informed by experience – the Yongbyon reactor, which has been shut down and restarted a number of times as a result of a series of agreements made (and subsequently broken) between North Korea and various negotiating bodies \cite{bajoria,davenport}, has been the source of material used in the nuclear weapons that North Korea has tested \cite{hecker}.

American and South Korean negotiators have thus taken the hard-line position that complete commitment to denculearization is a precondition to any negotiations with the North \cite{lee}. But this stance has proven unhelpful in getting North Korea to agree to negotiate \cite{kcna2}. They categorically refuse to denuclearize, claiming that it is their right as a sovereign nation to pursue both peaceful and military nuclear technology and that maintaining this right is a necessary deterrent against foreign aggression \cite{kcna,kcna2}. Each side, then, is demanding as a precondition for negotiation the very action that the other side refuses to even consider. It is perhaps unsurprising that this has not been effective.

The result has been a cycle of behavior that has gone on for several decades: aggression and posturing on North Korea’s part is followed by sanctions and demands by other nations. Tension escalates until one side or the other begins to call for negotiations, which either end in stalemate or produce results that only last for a short time before the agreements are broken again \cite{bajoria, davenport}.

Many analyses of the nature of the politics surrounding North Korean nuclear diplomacy exist, and several have concluded that the problem is cyclic in nature \cite{blair, cfr, fisher, gause, jun, habib}. In general, the issue tends to be modeled as a series of cycles comprised of initial posturing, followed by aggression and escalation, and concluding with reconciliation that, depending on the analyst, may or may not be identified as sincere.

When these analyses are politically-motivated \cite{blair, cfr}, their logic tends to begin with the initial proposition that demand for complete denuclearization can someday be fulfilled, if only the correct combination of diplomacy and coercion can be found that will force the DPRK to comply. More academic treatments of the problem \cite{habib,jun} vary widely in their approach. 

\section{Objective}

This thesis will combine analysis along three axes in order to practically judge North Korea's bargaining position. Examination of the historic record of negotiation proceedings will be combined with analysis of the rhetoric of press releases and official government statements, as well as calculations on nuclear material stockpiles, weapon testing yields, and production capacity. Unlike previous research, this combines qualitative and quantitative analysis in order to form a more complete picture of the North Korean bargaining position.

It is hypothesized that United States strategy in negotiating with North Korea is inconsistent with the technical facts of the North Korean nuclear program and past negotiating behavior. In other words, the cyclical nature of negotiations persists because negotiators base their positions on what they want to be true about North Korea, rather than the facts of the situation.

\section{Methodology}

Initial analysis will focus on the historic record with the intention of identifying specific patterns within negotiation cycles. Possible considerations include the timing and duration of various rounds of the Six-Party Talks, the timing and magnitude of sanctions and aid programs, the operational history of the Yongbyon reactor, and the dates and locations of missile and underground nuclear tests.

Analysis of the technical data available on North Korea’s nuclear capacity will begin from the results of historical analysis, but with more focus on the details of what occurred as opposed to when and where. In particular, information on underground nuclear tests will be coupled with data on what nuclear materials are known or suspected to have been shipped to the country. This will allow estimation both of the sorts of bombs North Korea is likely to possess, and of the material stockpiles they currently have access to.

The third major axis of analysis will be the examination of the rhetoric used in official government statements and press releases. Particular attention will be paid to the evolution of rhetoric over time – things like the phrases used by official North Korean news agencies to describe political adversaries, or which terms US officials emphasize when discussing previous agreements, are valuable indicators of general attitude. Looking at public statements in conjunction with historical data is worthwhile for determining when position statements are followed by action vs. when they turn out to be empty threats. Other information, like whether official statements are made by military officials or civilian leaders, may also be examined.

Conclusions will be drawn from the data gleaned from all three approaches. Though the end result cannot be determined at this point, a number of likely indicators of problems in American negotiation strategies can be predicted. If North Korea's technical achievements are in line with or on trajectory to match its stated end goals for a nuclear arsenal and infrastructure, it seems unlikely to acquiesce to external pressures to reverse course. If its rhetoric regarding denuclearization has been consistent across a wide variety of incentives and penalties, it is unlikely that variation in the terms of agreements along existing strategies will have a noticeable impact.

Meanwhile, there are also some indicators that American negotiation strategies hold the potential for success: if, for example, there have been previous agreements which, even if ultimately unsuccessful (as most if not all all have eventually been), provided some sort of incremental improvement to the situation that lasted beyond the agreement's end. Also, if North Korea's technical capacity is still far from their desired endpoint and seems unlikely to reach it for many years, there is likely to be more room for productive negotiations - provided that these negotiations take some form that is measurably different from methods that have been consistently unsuccessful in the past.


\section{Preliminary Results}
Initial historical analysis has occurred, providing insight into further directions for historic examination, and context for research along other axes. In particular, the concept of an "escalatory cycle" was defined\footnote{Escalatory cycle: a series of events consisting of a single initiating event and a number of subsequent actions, each in direct or explicit response to some previous action in the cycle.}, and ten such cycles were identified in the history of North Korean nuclear diplomacy.

Several trends were identified, namely: the relative success of diplomatic efforts initiated by North Korea as opposed to those initiated by external bodies; the gradual decrease in international trust in North Korea's professions of good-faith negotiating; and the apparent unwillingness of North Korean diplomats to contextualize their positions in terms of their previous actions.

Further work must still be done to ensure relevant details surrounding the various cycles have not been overlooked. Additionally, current events provide an interesting opportunity to validate these trends or at least add data to the cyclic model. The unfolding situation surrounding North Korean submarine missile testing certainly seems to fit the model of the escalatory cycle, but because of the timing of events with respect to when initial historical research occurred, it did not make sense to include in the first batch of analysis.

\section{Schedule}

\begin{center}
\begin{tabu} to 1.0\textwidth { | X[l] | X[r] | }
 \hline
 Task & Completion Time \\ 
 \hline
 \hline
 Qualitative research: historical trends & November 2015 \\ \hline
 Qualitative research: government statements/press releases & November 2015 \\
 \hline
 Quantitative research: materials acquisition and access & Early IAP (January 2016)\\
 \hline
 Quantitative research: enrichment/reprocessing capacity and likely stockpiles & End of IAP (February 2016)\\
 \hline
 Quantitative research: missiles, warhead miniaturization, threat radius & End of IAP (February 2016)\\
 \hline
 Analysis: North Korea’s bargaining position & February 2016\\
 \hline
 Analysis: comparison of bargaining position with international community’s behavior & February/March 2016\\
 \hline
 Full draft & Early April 2016\\
 \hline
 Complete thesis & Mid-April 2016\\
 \hline
 2-3 weeks leeway for everything to go wrong & Late April/Early May 2016\\
 \hline
 Thesis due & May 6, 2016\\
 \hline
\end{tabu}
\end{center}

\bibliographystyle{ieeetr}
\bibliography{prospectusreferences}

\end{document}
