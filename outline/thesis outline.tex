\documentclass{article}
\usepackage[utf8]{inputenc}
\usepackage{outlines}
\usepackage{enumitem}
\setlistdepth(5)
\setenumerate[1]{label=\Roman*.}
\setenumerate[2]{label=\Alph*.}
\setenumerate[3]{label=\roman*.}
\setenumerate[4]{label=\alph*.}
\setenumerate[5]{label=\Arabic*.}
\begin{document}
\begin{outline}[enumerate]


\1 Introduction
    \2 Motivation
        \3 	A nuclear-armed North Korea is concerning to a lot of people
            \4 \cite{kerry} (probably grab more cites)
            \4 In the past, there were thoughts that peaceful nuclear power could work, but in recent years this has been deemed unacceptable/too risky
            \4 Agreed Framework in 1994 planned to replace graphite reactors with light-water reactors \cite{agreed}
            \4 Foundation of Lee Myung-bak’s campaign in 2008 that South Korean aid could be dependent upon denuclearizing \cite{snyder}
            \4 US diplomatic position as of 2014 that guarantee of denuclearization key to resumption of 6-party talks \cite{panda}
        \3 The ways this has been dealt with historically haven’t worked
            \4 Temporary disarmament, but eventual return to status quo
            \4 \cite{blair,cfr,fisher,gause} [from prospectus, say the problem is cyclic]
            \4 \cite{davenport,nti15,iaea09} [each is a timeline]
            \4 Choose which kind of cites are appropriate here based on whether this ends up being more a “historical perspective” or a “people have identified this as a problem” kind of paragraph. Maybe both!
        \3 Maybe we should figure out why!

    \2 Objectives
        \3 Study technical capacity and stated objectives of North Korea to figure out whether current methods of diplomacy make sense
            \4 This doesn’t need to be super-detailed because it will be discussed in more depth in methodology section
            \4 But it should talk about how an empirical look at the problem may impart new insight
        \3 It may additionally be possible to move towards finding alternate methods of diplomacy to replace current failing methods
            \4 However, this is a lot more complicated and likely outside the scope of what can reasonably be accomplished
            \4 In which case this paragraph will be omitted

\1 Background
    \2 History of North Korean nuclear programs 
        \3 Nuclear power – Yongbyon reactor
            \4 Throughout the 1950s, North Korea and the USSR worked together to begin DPRK nuclear power program \cite{nti15}
            \4 Became party to some IAEA regulations in the 70s for their graphite reactor \cite{iaea09}
            \4 In 80s, continued to expand Yongbyon complex, look into LWR tech  \cite{ntiYongbyon}
            \4 Joined NPT in 85, left in 03 \cite{npt}
            \4 Agreed Framework between US and DPRK caused KEDO to occur \cite{agreed, davenport}, a politically-fraught set of agreements that was supposed to result in LWRs but eventually failed
            \4 Yongbyon keeps being shut down and opening again as a result of negotiations and proliferation concerns and IAEA inspectors getting kicked out and stuff \cite{davenport,nti15, iaea09}
        \3 Nuclear weapons – missile and underground nuclear testing
            \4 DPRK first major missile test in 98 \cite{orfall}
            \4 Again in 06, along with first nuclear test \cite{davenport}
            \4 To date, 3 nuclear tests and a number of missile tests \cite{davenport, nti15}
    \2 Attempts at negotiations
        \3 Discussion of 6-party talks and others, what agreements have come out of them, how long the agreements have been honored
        \3 (Large table of 6-party talks info goes here, hopefully in paragraph form instead of large table)
        \3 \cite{6pt, js4, js5, js6}
    \2 Previous analysis of the problem
        \3 Cycle of posturing, aggression, reconciliation (there are definitely some people who have formulated qualitative/descriptive models of this. talk about that here)
            \4 “the regime has actively engineered crises as a means to extract international aid in exchange for de-escalation, without making any substantive concessions that would have limited or reversed the development trajectory of its nuclear weapons programme” \cite{habib}
            \4 Habib \cite{habib} also believes this is inevitable because the nuclear program is key to maintenance of North Korean military-first ethos and will only ever be used as an insincere bargaining chip
            \4 Jun \cite{jun} actually identifies specific occurrences of the cycles (as I will talk about in historical trends section) and also identifies problems: talking past one another/demanding too much, reactive rather than proactive diplomacy, mistrust owing to too many repeats of the cycle
            
\1 Methodology
    \2 Analysis of historical trends
        \3 This section will look at what North Korea and relevant other parties have done in the past, and outline the cycles that are likely to emerge
        \3 Analysis of access and materials capacity
            \4 This section will look at how much material we know North Korea has, how much we suspect has entered/exited due to smuggling, what’s been used up in missile tests, and what remains for nukes.
            \4 Additionally, reprocessing capacity, missile technology, and nuke miniaturization will be examined
        \3 Analysis of government statements/press releases/proceedings of talks
            \4 This section will look at what negotiating parties have demanded, what’s actually come out of negotiations, and how governments talk about one another

\1 Historical analysis
    \2 Not going to copy/paste 5 pages of outline from  22.04 paper here, but what is going to go here is:
        \3 First, a timeline of events from the early 50s to the present, divided by identifiable cycles.
        \3 Then, set of conclusions that can be drawn strictly from the historic record, including:
            \4 North Korea acts as though it has no memory – every time it reacts outraged to international censure, this would be totally reasonable had it not aggressed first
            \4 When North Korea makes initial diplomatic overtures, general improvement to the situation occurs (that is, the cycle ends on a positive note – usually the resultant agreement gets broken to spur the next cycle, but there are temporary pleasantries)
            \4 On the other hand, when things get kicked off by external action (new sanctions, foreign policy changes, etc) North Korea tends to double down
            \4 Missile and nuclear tests tend to make things go poorly
            
\1 Technical Analysis
    \2 What technology does North Korea have access to?
        \3 Reprocessing
            \4 How quickly? How reliably?
            \4 Look at Hecker (the guy who was shown the reprocessing plant in the first place) for best info here, probably
            \4 This informs analysis in part (b) about how much bomb-making material is plausible
        \3 Missiles
            \4 Look at success and failure of recent tests
            \4 Photos of missiles on parade, claims by North Korean officials (These may be real but may also be a charade, examine expert opinions like 38north)
            \4 This informs North Korea’s credible threat range, if they can achieve miniaturization
        \3 Nukes, incl. miniaturization
            \4 How large have their tests been? What kinds of bombs?
            \4 Miniaturization is key to credible nuclear threats – how likely are they to have achieved it?
    \2 How much material does North Korea have, and of what kinds?
        \3 Known imports
            \4 Gather from initial Russian supply in 50s, provision under KEDO if any (I don’t think there was any)
        \3 Suspected smuggling imports
        \3 Suspected amount reprocessed
            \4 Look at number of known Yongbyon stoppages and fuel rod removals (probably from \cite{davenport}, which lists all the times North Korea has said it has reprocessed everything that was in Yongbyon, along with corroboration/completion from other sources) to gather initial ballpark estimate
            \4 Look at suspected removals and cross-reference with reprocessing capacity to get upper bound
            \4 Compare with existing numbers (State Department estimates, etc.) for validation
            \4 Don’t want to \emph{just} use those numbers because method of arrival at them likely to be classified or otherwise hard to verify
    \2 How much has been used in missile tests?
        \3 Best estimates come from seismological data and isotopic analysis from a distance after each missile test
        \3 Track down that data, get range of possible amounts
    \2 Thus, likely amount remaining to be weaponized

\1 Rhetorical analysis
    \2 Evolution of negotiating parties’ rhetoric over time
        \3 Look at KCNA coverage of each major crisis identified in Historical Analysis section
        \3 Look at State Department/presidential statements during same
        \3 (note: the outline of this section is short, because no analysis has been done yet - actual section will likely be quite lengthy)
    \2 Compare rhetoric to results
        \3 When treaties have been signed vs. when cycles have ended fruitlessly – is there some notable difference in how politicians talk about the problem?

\1 Results and Discussion
    \2 North Korea’s bargaining position
        \3 First, examine technical capacity
            \4 What do they currently have or can plausibly claim?
            \4 What would be lost if they were to sincerely give these things up? Militarily, and brief discussion of how that relates to internal politics as a whole
        \3 Then, examine historical analysis
            \4 What have they used nuclear threat for before?
            \4 They lose an effective bargaining chip – under what circumstances could that tradeoff be worthwhile?
        \3 Then, examine US/South Korean/international demands (especially US, but when a united front is presented that they’re part of it’s worthwhile to look at as well)
            \4 Are they taking into account the information identified above?
            \4 Does their negotiating strategy assume things that may or may not cohere with facts?

\1 Conclusions
    \2 Demands do/don’t make sense
    \2 Identify several points of possible mismatch, and whether they are overcome
    \2 It is/is not necessary for US negotiators to re-evaluate their priorities
    \2 X, Y, and Z must be taken into account
        \3 (if there are obvious things that could fill X/Y/Z here, they should be mentioned. If there are straightforward changes to the US diplomatic posture that could be made to account for these, they will be mentioned here. If there is nothing that can be straightforwardly identified, trying to do so is probably outside the scope of this thesis)

\end{outline}

\bibliographystyle{ieeetr}
\bibliography{outlinereferences}
\end{document}
